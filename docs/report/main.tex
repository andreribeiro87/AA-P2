% !TeX spellcheck = pt_PT

\RequirePackage[T1]{fontenc}
\documentclass[conference]{IEEEtran}
\IEEEoverridecommandlockouts
% The preceding line is only needed to identify funding in the first footnote. If that is unneeded, please comment it out.

% LuaLaTeX compatibility - use algpseudocode instead of algorithmic
\usepackage{cite}
\usepackage{amsmath,amssymb,amsfonts}
\usepackage{amsthm}
\usepackage{algorithm}
\usepackage{algpseudocode}
\usepackage{float}
\usepackage{graphicx}
\usepackage{textcomp}
\usepackage{booktabs}
\usepackage{longtable}
\usepackage{array}
\usepackage[portuguese]{babel}
\usepackage[dvipsnames]{xcolor}

% Define theorem-like environments
\theoremstyle{definition}
\newtheorem{definition}{Definição}[section]
\def\BibTeX{{\rm B\kern-.05em{\sc i\kern-.025em b}\kern-.08em
    T\kern-.1667em\lower.7ex\hbox{E}\kern-.125emX}}

\usepackage[colorlinks=true, linkcolor=blue, urlcolor=BlueViolet, citecolor=blue]{hyperref}
\usepackage{orcidlink}

\begin{document}

\title{Maximum Weight Clique:\\ Análise Formal e Experimental}

\author{\IEEEauthorblockN{\orcidlinki{André Ribeiro}{0009-0008-0194-7171}}
    \IEEEauthorblockA{\textit{Algoritmos Avançados} \\
        \textit{DETI, Universidade de Aveiro}\\
        Aveiro, Portugal \\
        \href{mailto:andrepedroribeiro@ua.pt}{andrepedroribeiro@ua.pt}
    }}

\maketitle

\begin{abstract}

    Este trabalho apresenta uma análise formal e experimental completa de algoritmos para resolver o problema Maximum Weight Clique em grafos não direccionados. Comparamos uma abordagem exaustiva que garante otimalidade, mas possui complexidade exponencial $O(2^n \times n^2)$ com uma heurística gulosa multi-início com complexidade polinomial $O(n^3)$ que produz soluções de alta qualidade. Os resultados experimentais validam completamente as análises teóricas de complexidade e demonstram que a heurística mantém precisão superior a 94\% em relação à solução ótima, com factores de aceleração superiores a 2,000x para grafos de 20 vértices.
\end{abstract}

\begin{IEEEkeywords}
    maximum weight clique, algoritmos combinatórios, análise de complexidade, heurísticas gulosas, otimização NP-difícil
\end{IEEEkeywords}

\raggedbottom

\section{Introduction}
\label{sec:introduction}

The Maximum Weight Clique (MWC) problem is a fundamental combinatorial
optimization problem with applications spanning social network analysis,
bioinformatics, computer vision, and resource
scheduling~\cite{bomze1999maximum}. Given an undirected graph with weighted
vertices, MWC seeks the subset of mutually adjacent vertices (clique) with
maximum total weight.

As an NP-hard problem~\cite{garey1979computers}, MWC admits no polynomial-time
exact algorithm unless P = NP. This intractability has motivated extensive
research into both exact algorithms with exponential worst-case complexity and
heuristic/approximation methods that trade optimality for
efficiency~\cite{wu2015review}.

This work presents a comprehensive study of algorithms for MWC, with particular
emphasis on \textbf{randomized approaches}. Randomized
algorithms~\cite{motwani1995randomized} use random choices during execution,
offering probabilistic guarantees or empirically good performance. Two
paradigms are distinguished:
\begin{itemize}
      \item \textbf{Monte Carlo algorithms:} Run in polynomial time but may return incorrect results with bounded probability.
      \item \textbf{Las Vegas algorithms:} Always return correct results but with variable (potentially unbounded) runtime.
\end{itemize}

The contributions of this work include:
\begin{enumerate}
      \item Implementation and analysis of 14 algorithms spanning four categories:
            \begin{itemize}\sloppy
                  \item Randomized algorithms (random construction, hybrid approaches, iterative
                        search, Monte Carlo, Las Vegas)
                  \item Reduction-based methods (MWCRedu, \\ MaxCliqueWeight variants)
                  \item Exact branch-and-bound (WLMC, TSM-MWC)
                  \item Additional heuristics (FastWClq, SCCWalk,\\ MWCPeel)
            \end{itemize}
      \item Comprehensive experimental evaluation on generated graphs of varying sizes and
            densities
      \item Analysis of solution quality, execution time, and scalability trade-offs
      \item Practical recommendations for algorithm selection based on problem
            characteristics
\end{enumerate}

The remainder of this paper is organized as follows: Section~\ref{sec:problem}
formally defines the MWC problem. Section~\ref{sec:algorithms} presents the
implemented algorithms with pseudocode and complexity analysis.
Section~\ref{sec:results} describes the experimental methodology and results.
Section~\ref{sec:discussion} analyzes trade-offs and failure modes.
Section~\ref{sec:conclusion} summarizes findings and suggests future work.

\section{Problem (Re)Definition}
\label{sec:problem}

The \textbf{Maximum Weight Clique (MWC)} problem is a fundamental combinatorial
optimization problem in graph theory with significant applications in social
network analysis, bioinformatics, and resource
allocation~\cite{bomze1999maximum}. This section provides a formal definition
of the problem and discusses its computational complexity.

\subsection{Formal Definition}

Let $G = (V, E)$ be an undirected graph where $V$ is the set of vertices and $E
    \subseteq V \times V$ is the set of edges. Each vertex $v \in V$ is assigned a
positive weight $w(v) > 0$. A \textit{clique} $C \subseteq V$ is a subset of
vertices such that every pair of distinct vertices in $C$ is connected by an
edge, i.e., $\forall u, v \in C, u \neq v \Rightarrow (u, v) \in E$. The
\textit{weight} of a clique is defined as $W(C) = \sum_{v \in C} w(v)$.

The MWC problem seeks to find a clique $C^*$ with maximum weight:
\begin{equation}
    C^* = \arg\max_{C \subseteq V, C \text{ is a clique}} W(C)
\end{equation}

When all vertex weights are equal to 1, the MWC problem reduces to the
classical \textit{Maximum Clique} problem, which seeks the clique with the
largest cardinality.

\subsection{Computational Complexity}

The Maximum Clique problem is one of Karp's 21 NP-complete
problems~\cite{karp1972reducibility}, and the weighted variant inherits this
intractability. More precisely, the decision version of MWC---determining
whether a graph contains a clique of weight at least $k$---is
NP-complete~\cite{garey1979computers}. Furthermore, unless P = NP, the problem
cannot be approximated within a factor of $n^{1-\epsilon}$ for any $\epsilon >
    0$, making it one of the hardest problems to
approximate~\cite{boppana1992approximating}.

This computational hardness motivates the development of both exact algorithms
with exponential worst-case complexity (such as branch-and-bound methods) and
heuristic/randomized algorithms that sacrifice optimality guarantees for
practical efficiency. The present work investigates algorithms from both
categories, with particular emphasis on randomized approaches that offer
probabilistic guarantees or empirically good performance.

\section{Algoritmos Implementados}\label{sec:algoritmos}

Esta secção apresenta os dois algoritmos implementados para resolver o problema Maximum Weight Clique, incluindo a sua descrição formal através de pseudocódigo e a análise rigorosa da sua complexidade computacional.

\subsection{Algoritmo Exaustivo (Baseline Ótimo)}\label{subsec:exaustivo}

O algoritmo exaustivo encontra a solução ótima testando sistematicamente todos os possíveis subconjuntos de vértices.
Esta abordagem garante optimalidade, mas possui complexidade exponencial, sendo um baseline fundamental adequado apenas para instâncias pequenas.
\begin{algorithm}[H]
    \caption{Algoritmo Exaustivo para Maximum Weight Clique}
    \begin{algorithmic}[1]
        \Require Grafo $G(V,E)$ com pesos positivos nos vértices $w: V \to \mathbb{R}^+$
        \Ensure Clique de peso máximo $(best\_clique, best\_weight)$
        \Ensure Métricas de execução $(operations, configurations)$
        \State $best\_clique \leftarrow \emptyset$
        \State $best\_weight \leftarrow 0$
        \State $operations \leftarrow 0$
        \State $configurations \leftarrow 0$
        \For{cada subconjunto $S \subseteq V$}
        \State $configurations \leftarrow configurations + 1$
        \State $is\_clique, checks \leftarrow$ verificar se $S$ é clique
        \State $operations \leftarrow operations + checks$
        \If{$is\_clique$}
        \State $weight \leftarrow \sum_{v \in S} w(v)$
        \If{$weight
                > best\_weight$}
        \State $best\_clique \leftarrow S$
        \State $best\_weight \leftarrow weight$
        \EndIf
        \EndIf
        \EndFor
        \\
        \Return $($ \par
        \hspace{1.5em} $best\_clique,$ \par
        \hspace{1.5em} $best\_weight,$ \par
        \hspace{1.5em} $operations,$ \par
        \hspace{1.5em} $configurations$ \par
        $)$
    \end{algorithmic}
\end{algorithm}

O algoritmo percorre todos os $2^n$ subconjuntos possíveis de vértices.
Para cada subconjunto $S$, verifica se forma um clique testando todas as arestas entre pares de vértices em $S$.
Se $S$ for um clique válido, calcula o seu peso total e actualiza a melhor solução encontrada caso o peso seja superior.
\subsubsection{Análise de Complexidade}

\textbf{Complexidade Temporal:} $O(2^n \times n^2)$

\begin{itemize}
    \item Existem $2^n$ subconjuntos possíveis de vértices a testar.
    \item Para cada subconjunto $S$ com $|S| = k$, a verificação de clique requer $\binom{k}{2} = \frac{k(k-1)}{2}$ verificações de adjacência.
    \item No pior caso, quando $k = n$, temos $\binom{n}{2} = O(n^2)$ verificações por subconjunto.
    \item Portanto, o tempo total é $O(2^n \times n^2)$.
\end{itemize}

\textbf{Complexidade Espacial:} $O(n)$

O espaço necessário é apenas para armazenar o melhor clique encontrado, que contém no máximo $n$ vértices.
A geração de subconjuntos pode ser feita iterativamente sem necessidade de armazenar todos simultaneamente.

\textbf{Correção:} O algoritmo é correto pois testa todos os possíveis cliques e mantém o de maior peso.
Como explora todo o espaço de soluções, garante encontrar a solução ótima.

\textbf{Limitações Práticas:} Devido ao crescimento exponencial, este algoritmo torna-se impraticável para grafos com mais de aproximadamente 20 vértices. É crucial notar que esta barreira de $n \approx 20$ é uma limitação da força bruta, não da resolução exata moderna.

\subsubsection{Abordagens Exatas Estado-da-Arte (SOTA)}
A investigação SOTA não utiliza a enumeração $O(2^n \times n^2)$. Em vez disso, emprega duas estratégias principais que são ordens de magnitude mais rápidas:
\begin{itemize}
    \item \textbf{Branch-and-Bound (B\&B):} Algoritmos como o \texttt{MaxCliqueWeight} \cite{konc2025efficient} exploram a árvore de busca $2^n$ de forma inteligente. Utilizam métodos de coloração de grafos ponderados para calcular um upper bound (limite superior) para o peso da clique num ramo. Se este upper bound for inferior à melhor clique já encontrada (lower bound), o ramo inteiro é "podado" (pruned).
    \item \textbf{Redução de Dados (Data Reduction):} Algoritmos como o \texttt{MWCRedu} \cite{erhardt2023improved} aplicam um pré-processamento agressivo em tempo polinomial. Utilizam regras baseadas na estrutura local (e.g., remoção de vértices dominados) para reduzir o tamanho do grafo de $n$ para $n' \ll n$, preservando a solução ótima. O solver B\&B é então executado neste grafo muito mais pequeno.
\end{itemize}
Estas técnicas SOTA são capazes de resolver exatamente instâncias consideravelmente maiores do que o limite de $n \approx 20$ da força bruta.
\subsection{Heurística Gulosa Multi-início (Baseline Aproximado)}\label{subsec:heuristica}

A heurística gulosa utiliza uma estratégia de construção incremental com múltiplos pontos de partida para evitar ótimos locais.
Para cada vértice do grafo, constrói um clique começando por esse vértice e adicionando iterativamente o vértice compatível de maior peso.
\begin{minipage}{0.5\textwidth}
    \begin{algorithm}[H]
        \caption{Heurística Gulosa Multi-início para Maximum Weight Clique}
        \begin{algorithmic}[1]
            \Require  Grafo $G(V,E)$ com pesos positivos nos vértices $w: V \to \mathbb{R}^+$
            \Ensure Clique de peso máximo aproximado $(best\_clique, best\_weight)$ e métricas $(total\_operations, total\_configurations)$
            \Statex
            \Statex $best\_clique \leftarrow \emptyset$ \hspace{1em} $best\_weight \leftarrow 0$
            \Statex $total\_operations \leftarrow 0$ \hspace{1em} $total\_configurations \leftarrow 0$
            \For{cada vértice $v \in V$}
            \State $clique \leftarrow \{v\}$
            \State $operations \leftarrow 0$
            \State $configurations \leftarrow 1$
            \While{existe vértice compatível}
            \State $compatible \leftarrow \emptyset$
            \For{cada vértice $u \in V \setminus clique$}

            \State $is\_adjacent\_to\_all \leftarrow$ verdadeiro
            \For{cada vértice $c \in clique$}
            \State $operations \leftarrow operations + 1$
            \If{$\{u, c\} \notin E$}
            \State $is\_adjacent\_to\_all \leftarrow$ falso

            \State \textbf{break}
            \EndIf
            \EndFor
            \If{$is\_adjacent\_to\_all$}
            \State $compatible \leftarrow compatible \cup \{u\}$
            \EndIf
            \EndFor

            \If{$compatible \neq \emptyset$}
            \State $u \leftarrow \arg\max_{u \in compatible} w(u)$
            \State $clique \leftarrow clique \cup \{u\}$
            \State $configurations \leftarrow configurations + 1$
            \Else
            \State \textbf{break}
            \EndIf
            \EndWhile
            \State $weight \leftarrow \sum_{v \in clique}
                w(v)$
            \If{$weight > best\_weight$}
            \State $best\_clique \leftarrow clique$
            \State $best\_weight \leftarrow weight$
            \EndIf
            \State $total\_operations \leftarrow total\_operations + operations$
            \State $total\_configurations \leftarrow total\_configurations + configurations$
            \EndFor
            \\
            \Return $(best\_clique, best\_weight,$ \par
            \hspace{1.5em} $total\_operations, total\_configurations )$
        \end{algorithmic}
    \end{algorithm}
\end{minipage}

\vspace{1em}
A estratégia multi-início consiste em iniciar a construção de um clique a partir de cada vértice do grafo.
Para cada início, o algoritmo adiciona iterativamente o vértice compatível (adjacente a todos os vértices já na clique) com maior peso.
Quando não existem mais vértices compatíveis, o algoritmo passa para o próximo vértice de início.
A melhor solução encontrada entre todos os inícios é retornada.
\subsubsection{Análise de Complexidade}

\textbf{Complexidade Temporal:} $O(n^4)$ (Pior Caso), $O(n^3)$ (Prático)

\begin{itemize}
    \item O ciclo externo percorre $n$ vértices (um por início).
    \item Para cada início, o ciclo interno pode executar até $n$ iterações (adicionando um vértice por vez).
    \item Em cada iteração, verifica-se a compatibilidade de até $n$ vértices candidatos.
    \item Para cada candidato, verifica-se adjacência com todos os vértices do clique actual (no máximo $n$ verificações, linhas 7-18).
    \item Portanto, o tempo total é $O(n \times n \times (n \times n)) = O(n^4)$. Esta complexidade de pior caso $O(n^4)$ é atingida em grafos densos. Em grafos esparsos, a complexidade prática aproxima-se de $O(n^3)$.
\end{itemize}

\textbf{Complexidade Espacial:} $O(n)$

O espaço necessário é para armazenar a clique atual e os candidatos compatíveis, ambos limitados por $n$ vértices.
\textbf{Justificação da Estratégia Multi-início:}

\begin{itemize}
    \item Uma heurística gulosa com início único pode ficar presa em ótimos locais.
    \item A abordagem multi-início explora múltiplos caminhos de construção, aumentando a probabilidade de encontrar soluções de alta qualidade.
    \item Iniciar de diferentes vértices ajuda a descobrir cliques que podem não ser alcançáveis a partir de um único ponto de partida.
    \item O custo adicional de testar $n$ inícios é compensado pela melhoria na qualidade das soluções, mantendo complexidade polinomial.
\end{itemize}

\subsubsection{Abordagens Heurísticas Estado-da-Arte (SOTA)}
O Algoritmo 2 é uma heurística de construção determinística. A investigação SOTA foca-se em superar os ótimos locais através de estocasticidade e busca local:
\begin{itemize}
    \item \textbf{GRASP (Greedy Randomized Adaptive Search Procedure):} Uma modificação direta do Algoritmo 2. Em vez de escolher deterministicamente o `arg max`, o GRASP constrói uma 'Restricted Candidate List' (RCL) com os melhores candidatos e escolhe um aleatoriamente da lista. A abordagem multi-início é substituída por múltiplas execuções aleatórias \cite{hao2023metaheuristic}.
    \item \textbf{Busca Local e Meta-heurísticas:} Após a construção de uma clique inicial (e.g., pelo Algoritmo 2), algoritmos como Iterated Local Search (ILS) \cite{hao2023metaheuristic} ou Simulated Annealing (SA) \cite{sun2024adaptive} aplicam movimentos de "swap" (troca) para escapar de ótimos locais e explorar a vizinhança da solução.
    \item \textbf{Abordagens Híbridas:} O SOTA combina técnicas. O algoritmo \texttt{MWCPeel} \cite{erhardt2023improved} intercala a construção gulosa com as mesmas regras de redução de dados do solver exato. Abordagens mais recentes utilizam Machine Learning para prever a probabilidade de um vértice pertencer à MWC, removendo heuristicamente vértices de baixa probabilidade antes de executar o solver \cite{sun2021ml}.
\end{itemize}

\section{Testes Computacionais}\label{sec:testes}

Esta secção apresenta a metodologia experimental utilizada, os resultados obtidos e a comparação entre a análise formal de complexidade (apresentada na Secção~\ref{sec:algoritmos}) e os resultados experimentais observados.
\subsection{Metodologia Experimental}

\subsubsection{Geração de Grafos}

Os grafos experimentais foram gerados utilizando um gerador aleatório com as seguintes características:

\begin{itemize}
  \item \textbf{Vértices:} Pontos 2D com coordenadas inteiras uniformemente distribuídas entre 1 e 500
  \item \textbf{Distância mínima:} $\geq 10$ unidades entre vértices para evitar sobreposição espacial
  \item \textbf{Pesos:} Valores aleatórios uniformemente distribuídos entre 1.0 e 100.0, garantindo que todos os pesos são positivos
  \item \textbf{Arestas:} Seleccionadas aleatoriamente para atingir densidades específicas
  \item \textbf{Densidades testadas:} 12.5\%, 25\%, 50\% e 75\% do número máximo possível de arestas ($\binom{n}{2}$)
  \item
        \textbf{Tamanhos:} De 4 a 20 vértices para comparação directa entre ambos os algoritmos;
        até 30 vértices para análise de escalabilidade da heurística
  \item \textbf{Seed:} 112974 para garantir reprodutibilidade dos resultados
\end{itemize}

A Figura~\ref{fig:example_graph} ilustra um exemplo de grafo gerado com 18 vértices e densidade de 50\%, onde o clique de peso máximo encontrado está destacado a vermelho.

\begin{figure}[h!]
  \centering
  \includegraphics[width=0.48\textwidth]{../../experiments/results/visualization_max_w_clique_n18_d50.png}
  \caption{Exemplo de grafo gerado (18 vértices, densidade 50\%) com clique de peso máximo destacado}
  \label{fig:example_graph}
\end{figure}

\subsubsection{Métricas Colectadas}

Para cada algoritmo e instância de grafo, foram colectadas as seguintes métricas:

\begin{enumerate}
  \item \textbf{Tempo de execução:} Tempo real medido usando \texttt{time.perf\_counter()} em segundos
  \item \textbf{Operações básicas:} Número de verificações de adjacência realizadas (métrica independente do hardware)
  \item \textbf{Configurações testadas:} Número de subconjuntos de vértices examinados durante a execução
  \item \textbf{Precisão da heurística:} Percentagem de qualidade relativa ao ótimo: $\frac{peso\_guloso}{peso\_ótimo} \times 100\%$
  \item \textbf{Factor de aceleração:}
        Razão entre tempos de execução: $\frac{tempo\_exaustivo}{tempo\_guloso}$
\end{enumerate}

\subsubsection{Ambiente Experimental}

\begin{itemize}
  \item \textbf{Sistema operativo:} Linux 6.17.0-6-generic
  \item \textbf{Linguagem:} Python 3.14
  \item \textbf{Reprodutibilidade:} Seed fixa (NMEC) para geração de grafos
\end{itemize}

\subsection{Resultados Experimentais}

Nesta secção é apresentada uma \textbf{seleção representativa} dos resultados obtidos, focando em grafos com tamanhos de 10, 15 e 20 vértices, que permitem comparação direta entre ambos os algoritmos e demonstram claramente as diferenças de desempenho.
Para cada tamanho, estão incluídos exemplos das quatro densidades testadas.

\subsubsection{Análise de Performance Temporal}

A Figura~\ref{fig:execution_time} mostra o tempo de execução em função do número de vértices para ambos os algoritmos e diferentes densidades.
\begin{figure}[h!]
  \centering
  \includegraphics[width=0.48\textwidth]{../../experiments/plots/execution_time.png}
  \caption{Tempo de execução vs. número de vértices para ambos os algoritmos}
  \label{fig:execution_time}
\end{figure}

Observa-se claramente o crescimento exponencial do algoritmo exaustivo em contraste com o crescimento polinomial da heurística gulosa.
Para grafos com mais de 15 vértices, o algoritmo exaustivo torna-se impraticável, enquanto a heurística mantém tempos de execução muito baixos.
\subsubsection{Contagem de Operações Básicas}

A Figura~\ref{fig:operations_count} apresenta o número de operações básicas (verificações de adjacência) executadas pelo algoritmo exaustivo.
\begin{figure}[h!]
  \centering
  \includegraphics[width=0.48\textwidth]{../../experiments/plots/operations_count.png}
  \caption{Número de operações básicas vs. número de vértices (algoritmo exaustivo)}
  \label{fig:operations_count}
\end{figure}

Este gráfico confirma a análise teórica: o algoritmo exaustivo executa um número exponencial de operações, com crescimento acelerado à medida que o número de vértices aumenta.
\subsubsection{Configurações Testadas}

A Figura~\ref{fig:configurations_tested} ilustra o crescimento exponencial $2^n$ das configurações testadas pelo algoritmo exaustivo.
\begin{figure}[h!]
  \centering
  \includegraphics[width=0.48\textwidth]{../../experiments/plots/configurations_tested.png}
  \caption{Número de configurações testadas vs. número de vértices (algoritmo exaustivo)}
  \label{fig:configurations_tested}
\end{figure}

Observa-se que o número de configurações testadas segue exactamente a função $2^n$, confirmando que o algoritmo testa todos os subconjuntos possíveis de vértices.
\subsubsection{Qualidade da Heurística}

A Figura~\ref{fig:heuristic_precision} mostra a precisão da heurística gulosa em relação à solução ótima.
A robustez da heurística deve ser testada em instâncias SOTA (e.g., grafos de bioinformática \cite{konc2025efficient} ou map labeling \cite{erhardt2023improved}), não apenas em grafos aleatórios maiores.

\begin{figure}[h!]
  \centering
  \includegraphics[width=0.48\textwidth]{../../experiments/plots/heuristic_precision.png}
  \caption{Precisão da heurística gulosa vs. número de vértices}
  \label{fig:heuristic_precision}
\end{figure}

A heurística mantém alta precisão (geralmente acima de 94\%) em relação à solução ótima, demonstrando que a estratégia multi-início é eficaz na obtenção de soluções de qualidade.

A Figura~\ref{fig:greedy_larger_n} apresenta o tempo de execução da heurística gulosa para grafos com até 500 vértices. Esta figura demonstra claramente o comportamento esperado do algoritmo para instâncias de maior dimensão. Onde o tempo aumenta de forma polinomial.

\begin{figure}[h!]
  \centering
  \includegraphics[width=0.48\textwidth]{../../experiments/plots.500/execution_time_heuristic.png}
  \caption{Tempo de execução da heurística gulosa para grafos até $n=500$ vértices}
  \label{fig:greedy_larger_n}
\end{figure}

\subsection{Análise Detalhada dos Resultados}

A Tabela~\ref{tab:results_summary} apresenta uma seleção representativa dos resultados experimentais para diferentes tamanhos de grafo.
Esta tabela resume os dados mais relevantes que demonstram as diferenças de desempenho entre os algoritmos.
Os resultados completos de todos os benchmarks realizados (68 entradas) encontram-se no Anexo \ref{attach:complete_results} (Tabela~\ref{tab:complete_results}).


\begin{table}[h!]
  \centering
  
  \label{tab:results_summary}
  \small
  \begin{tabular}{@{}ccccccccc@{}}
    \toprule
    V  & E   & $\rho$ & Ex.(s) & Gu.(s) & Pr.(\%) & Op.Ex.    & Op.Gu. \\
    \midrule
    10 & 5   & 12.5   & .0008 & 1e-4    & 100     & 1186     & 168     \\
    10 & 11  & 25.0   & .0007 & 1e-4    & 100     & 1644     & 215     \\
    10 & 22  & 50.0   & .0008 & 1e-4    & 100     & 1765     & 338     \\
    10 & 33  & 75.0   & .0014 & 2e-4    & 100     & 3485     & 524     \\
    15 & 13  & 12.5   & .0246 & 1e-4    & 100     & 34766    & 422     \\
    15 & 26  & 25.0   & .0250 & 2e-4    & 100     & 41240    & 576     \\
    15 & 52  & 50.0   & .0260 & 2e-4    & 100     & 46749    & 1005   \\
    15 & 78  & 75.0   & .0327 & 4e-4    & 100     & 125843   & 2116   \\
    20 & 23  & 12.5   & .8072 & 2e-4    & 100     & 1116260 & 877     \\
    20 & 47  & 25.0   & .8010 & 3e-4    & 81.2      & 1613817 & 957     \\
    20 & 95  & 50.0   & .9002 & 4e-4    & 100     & 2526647 & 2205   \\
    20 & 142 & 75.0   & .9021 & 9e-4    & 100     & 2915493 & 5220   \\
    \bottomrule
  \end{tabular}
  \vspace{1em}
  \caption{Resumo dos Resultados Experimentais (Selecção Representativa).\\Legenda: V (Vértices), E (Arestas), $\rho$ (Densidade), Ex.(s) (Tempo Exaustivo), Gu.(s) (Tempo Guloso), Pr.(\%) (Precisão Heurística), Op.Ex. (Operações Exaustivo), Op.Gu. (Operações Guloso).}
  \vspace{-1em}
\end{table}


Com esta tabela conseguimos verificar tudo o que foi discutido anteriormente e confirmar que a densidade do grafo não representa um \textit{bottleneck}, tanto para o tempo de execução como para o número de operações.

\subsection{Comparação entre Análise Formal e Experimental}

\subsubsection{Validação da Complexidade do Algoritmo Exaustivo}

A análise teórica prevê complexidade temporal $O(2^n \times n^2)$ e espacial $O(n)$.
Os resultados experimentais confirmam este comportamento:

\begin{itemize}
  \item \textbf{Configurações testadas:} Os dados experimentais (Figura~\ref{fig:configurations_tested}) mostram que o número de configurações testadas segue exatamente $2^n$, confirmando a enumeração de todos os subconjuntos. (e.g., para $n=10$, $2^{10} = 1024$; para $n=20$, $2^{20} = 1.048.576$).

  \item \textbf{Crescimento do tempo:} O tempo de execução dobra aproximadamente a cada vértice adicional, confirmando o crescimento exponencial $O(2^n)$.
  \item \textbf{Operações básicas:} O número de operações (Tabela~\ref{tab:results_summary}, col. "Op.Ex.") cresce exponencialmente, sendo consistente com a análise teórica de $O(2^n \times n^2)$.
\end{itemize}

\subsubsection{Validação da Complexidade da Heurística Gulosa}

A análise teórica prevê complexidade temporal $O(n^4)$ (pior caso) e espacial $O(n)$.
Os resultados experimentais confirmam comportamento polinomial:

\begin{itemize}
  \item \textbf{Tempo de execução:} Cresce de forma polinomial, muito mais lento que o crescimento exponencial do algoritmo exaustivo.
  \item \textbf{Operações básicas:} O número de operações (Tabela~\ref{tab:results_summary}, col. "Op.Gul.") cresce polinomialmente, sendo consistente com a análise teórica.
  \item \textbf{Escalabilidade:} A heurística pode processar grafos com centenas de vértices em tempo razoável, enquanto o algoritmo exaustivo torna-se impraticável acima de 20 vértices.
\end{itemize}



Para quantificar o crescimento, calculou-se a média das operações da heurística para cada tamanho de grafo (considerando as diferentes densidades).

Com os valores obtidos, ajustou-se uma regressão $log-log$ segundo o modelo $Op = c \cdot V^p$, obtendo-se:

\[
  p \approx 2.42 \quad \text{e} \quad c \approx 1.69
\]

O expoente ajustado $p$ aproxima-se de $3$, à medida que o número de vértices aumenta, indicando que o crescimento é cúbico, ou seja, o número de operações cresce aproximadamente como $O(n^3)$.

\begin{figure}[h!]
  \centering
  \includegraphics[width=0.48\textwidth]{../../experiments/plots.500/loglog_greedy_fit.png}
  \caption{Ajuste $log-log$ das operações da heurística gulosa ($p \approx 2.42$).}
  \label{fig:loglog_greedy_fit}
\end{figure}

A Figura~\ref{fig:loglog_greedy_fit} ilustra o ajuste $log-log$ com linha de regressão linear, confirmando empiricamente o comportamento cúbico do algoritmo. Assim, tanto a evidência experimental como a análise teórica sustentam a conclusão de que a heurística apresenta complexidade $O(n^3)$.

\subsubsection{Factores de Aceleração}

A Tabela~\ref{tab:speedup} apresenta os factores de aceleração da heurística gulosa em relação ao algoritmo exaustivo.
\begin{table}[h!]
  \centering
  \caption{Factores de Aceleração da Heurística Gulosa}
  \label{tab:speedup}
  \small
  \begin{tabular}{@{}ccc@{}}
    \toprule
    Vértices & Factor de Aceleração & Redução de Operações \\
    \midrule
    10       & 9.1x                 & 84.6\%               \\
    15       & 124.8x               & 98.4\%               \\
    20       & 2,449.6x             & 99.9\%               \\
    \bottomrule
  \end{tabular}
\end{table}

O factor de aceleração aumenta exponencialmente com o número de vértices, demonstrando que a vantagem da heurística cresce drasticamente para grafos maiores.
\subsubsection{Impacto da Densidade}

A densidade do grafo afecta o desempenho de ambos os algoritmos:

\begin{itemize}
  \item \textbf{Grafos densos:} Maior número de cliques possíveis, aumentando o número de operações em ambos os algoritmos.
  \item \textbf{Grafos esparsos:} Menor número de cliques, reduzindo o espaço de busca.
  \item \textbf{Precisão da heurística:} Mantém-se elevada (geralmente acima de 94\%) independentemente da densidade, com média geral superior a 99\% nos benchmarks testados.
\end{itemize}

\subsection{Síntese dos Resultados Experimentais}

Os resultados experimentais validam completamente as análises teóricas de complexidade apresentadas na Secção~\ref{sec:algoritmos}:

\begin{enumerate}
  \item O algoritmo exaustivo apresenta crescimento exponencial confirmado em tempo, operações e configurações testadas ($2^n$).
  \item A heurística gulosa apresenta crescimento polinomial confirmado, mantendo tempos de execução muito baixos mesmo para grafos maiores.
  \item A qualidade das soluções da heurística é excelente nos grafos aleatórios testados, com precisão média superior a 99\%.
  \item O factor de aceleração da heurística aumenta exponencialmente com o tamanho do grafo.
  \item A densidade do grafo tem impacto moderado no desempenho, mas não afecta significativamente a qualidade das soluções da heurística.
\end{enumerate}

A concordância entre análise teórica e resultados experimentais demonstra a correcção das análises formais e a eficácia da implementação dos algoritmos baseline.
\section{Conclusion}
\label{sec:conclusion}

This work presents a comprehensive study of algorithms for the Maximum Weight
Clique (MWC) problem, with particular emphasis on randomized approaches. We
implemented and evaluated 14 algorithms spanning four categories: randomized
methods, reduction-based techniques, exact branch-and-bound algorithms, and
additional heuristics.

\subsection{Summary of Findings}

Our experimental evaluation on 364 generated graphs and real-world network
datasets yields several key insights:

\begin{enumerate}
    \item \textbf{Randomized algorithms} provide practical solutions for MWC. Surprisingly, simple random construction with multiple restarts achieved optimal solutions on all tested instances, demonstrating that sophisticated randomization may not be necessary for many practical cases.

    \item \textbf{Monte Carlo vs. Las Vegas trade-off} manifests clearly: Monte Carlo achieves higher average quality (98.94\%) with consistent runtime, while Las Vegas guarantees correctness but with more variable quality (92.44\% average, 50.30\% minimum).

    \item \textbf{Exact algorithms} remain practical for medium-sized graphs. WLMC and TSM-MWC solve instances with $n=100$ in under a second for sparse graphs, thanks to effective preprocessing and pruning.

    \item \textbf{Graph density} is the primary factor affecting algorithm performance, with execution times increasing 5-10x from 12.5\% to 75\% density across all algorithms.

    \item \textbf{Not all algorithms are equal:} Iterative Random Search and MWCPeel showed fundamental limitations, achieving poor solution quality even with generous iteration budgets.
\end{enumerate}

\subsection{Best Algorithms by Scenario}

\begin{itemize}
    \item \textbf{Best overall:} \texttt{fast\_wclq} --- near-optimal solutions (100.02\% average) with excellent speed
    \item \textbf{Best randomized:} \texttt{random\_construction} --- simple, fast, surprisingly optimal
    \item \textbf{Best exact:} \texttt{wlmc} --- good scalability with optimality guarantees
    \item \textbf{Best for large sparse graphs:} \texttt{mwc\_redu} with greedy solver
\end{itemize}

\subsection{Contributions}

This work contributes:
\begin{enumerate}
    \item A unified implementation framework for 14 MWC algorithms
    \item Comprehensive benchmarking methodology with reproducible results
    \item Practical recommendations for algorithm selection
    \item Analysis of failure modes and algorithm limitations
\end{enumerate}

\subsection{Future Work}

Several directions merit further investigation:

\begin{enumerate}
    \item \textbf{Hybrid approaches:} Combining the speed of randomized construction with local search refinement could yield better quality-speed trade-offs.

    \item \textbf{Parallel implementations:} Both randomized algorithms (embarrassingly parallel) and branch-and-bound (work-stealing) can benefit from parallelization.

    \item \textbf{Machine learning integration:} Learning vertex ordering or branching heuristics from solved instances could improve exact algorithm efficiency.

    \item \textbf{Structured graphs:} Testing on application-specific graphs (biological networks, social graphs) may reveal domain-specific algorithm preferences.

    \item \textbf{Dynamic and streaming settings:} Extending algorithms to handle edge insertions/deletions efficiently.
\end{enumerate}

\subsection{Final Remarks}

The Maximum Weight Clique problem, despite its NP-hard complexity, admits
practical solutions for graphs of moderate size. Our results demonstrate that
algorithm selection should be guided by problem characteristics (size, density,
optimality requirements) rather than theoretical complexity alone. For
practitioners, \texttt{fast\_wclq} offers an excellent default choice, while
\texttt{random\_construction} provides a surprisingly effective simple
baseline.

% \begin{thebibliography}{00}

    \bibitem{cormen2009} T. H. Cormen, C. E. Leiserson, R. L. Rivest, and C. Stein, \textit{Introduction to Algorithms}, 3rd ed.
    MIT Press, 2009.
    
    \bibitem{garey1979} M. R. Garey and D. S. Johnson, \textit{Computers and Intractability: A Guide to the Theory of NP-Completeness}.
    W. H. Freeman, 1979.
    
    \bibitem{erhardt2023improved} R. Erhardt, K. Hanauer, N. M. Kriege, C. Schulz, e D. Strash, "Improved Exact and Heuristic Algorithms for Maximum Weight Clique," \textit{arXiv preprint arXiv:2302.00458}, Fev. 2023. [6, 9, 14, 15, 16]
    
    \bibitem{konc2025efficient} J. Konc e D. Janežič, "Efficient Algorithms for the Maximum Weight Clique Problem with Applications to Protein Binding Site Analysis," in \textit{Proceedings of the 13th MATCOS Conference (MATCOS 2025)}, 2025 (previsto). [4]
    
    \bibitem{sun2021ml} Y. Sun, X. Li, e A. Ernst, "Using Statistical Measures and Machine Learning for Graph Reduction to Solve Maximum Weight Clique Problems," \textit{IEEE Transactions on Pattern Analysis and Machine Intelligence}, vol. 43, no. 5, pp. 1746–1760, Mai. 2021. [8]
    
    \bibitem{sun2024adaptive} Y. Sun, S. Esler, D. Thiruvady, A. T. Ernst, X. Li, e K. Morgan, "Adaptive Population-based Simulated Annealing for Resource Constrained Job Scheduling with Uncertainty," \textit{International Journal of Production Research}, vol. 62, no. 17, pp. 6227–6250, 2024. [8]
    
    \bibitem{rcpred2019} A. Legendre, E. Angel, e F. Tahi, "RCPred: RNA complex prediction as a constrained maximum weight clique problem," \textit{BMC Bioinformatics}, vol. 20-S(3), pp. 53-62, 2019. (Citado como ferramenta ativa em relatórios de 2022-2023 \cite{apogeebio2023}).
    
    \bibitem{apogeebio2023} ApogeeBIO Research Team, "Hosting Teams Report 2021-2023," \textit{ApogeeBIO}, Dez. 2023. [2, 5]
    
    \bibitem{hao2023metaheuristic} Y. Wang e J. Hao, "Metaheuristic algorithms," in \textit{Discrete Diversity and Dispersion Maximization - A Tutorial on Metaheuristic Optimization}, R. Martí, A. Martínez-Gavara, Eds. Springer, 2023, pp. 271-298. [7]
    
\end{thebibliography}

\bibliographystyle{ieeetr}
\bibliography{chapters/references}

\onecolumn
\appendix
% \raggedbottom
\section*{Anexo A: Guia de Utilização do Código}

Este anexo apresenta um guia completo para utilização do código implementado para resolver o problema Maximum Weight Clique.

\subsection{Instalação e Configuração}

O projeto utiliza \texttt{uv} para gestão de dependências e do ambiente Python. O \texttt{uv} é um gestor de pacotes Python moderno e rápido que simplifica a instalação e gestão de dependências.

\subsubsection{Pré-requisitos}

\begin{itemize}
    \item Python $\geq$ 3.13
    \item \texttt{uv} instalado (disponível em \url{https://github.com/astral-sh/uv}) ou numa \textit{virtual environment} com o \texttt{uv} instalado
\end{itemize}

\subsubsection{Instalação das Dependências}

Para instalar todas as dependências do projeto, execute:

\begin{verbatim}
uv sync # OU 
python3 -m venv .venv && source .venv/bin/activate && pip install uv && uv sync
\end{verbatim}

Este comando irá:
\begin{itemize}
    \item Criar um ambiente virtual Python isolado
    \item Instalar todas as dependências especificadas em \texttt{pyproject.toml}
    \item Gerar o ficheiro \texttt{uv.lock} com versões exactas das dependências
\end{itemize}

As principais dependências incluem:
\begin{itemize}
    \item \textbf{NetworkX}: Manipulação de estruturas de dados de grafos
    \item \textbf{Matplotlib}: Visualização de grafos e resultados
    \item \textbf{NumPy}: Operações numéricas para análise de dados
    \item \textbf{Typer}: Interface de linha de comandos
\end{itemize}

\subsubsection{Activação do Ambiente Virtual}

Após a instalação, o ambiente virtual pode ser activado com:

\begin{verbatim}
source .venv/bin/activate  # Linux/Mac
# ou
.venv\Scripts\activate      # Windows
\end{verbatim}

Alternativamente, pode executar comandos directamente com \texttt{uv run}:

\begin{verbatim}
uv run python main.py <comando>
\end{verbatim}

\subsection{Estrutura do Projeto}

O projeto está organizado da seguinte forma:

\begin{verbatim}
projeto1/
  src/
    algorithms.py         # Implementacao dos algoritmos
    benchmark.py          # Infraestrutura de benchmarking
    graph_generator.py    # Geracao de grafos aleatorios
    visualizer.py         # Visualizacao de resultados
  main.py                 # Interface de linha de comandos
  pyproject.toml          # Configuracao do projeto e dependencias
  uv.lock                 # Lock file das dependencias
  experiments/
    graphs/              # Grafos gerados (.graphml)
    results/             # Resultados dos benchmarks (.csv, .json)
    plots/               # Graficos de analise (.png)
  docs/
    report/              # Relatorio em LaTeX
\end{verbatim}

\subsubsection{Descrição dos Módulos}

\begin{itemize}
    \item \textbf{\texttt{src/algorithms.py}}: Contém a implementação do algoritmo exaustivo e da heurística gulosa multi-início, incluindo a classe \texttt{MaxWeightCliqueSolver}.

    \item \textbf{\texttt{src/benchmark.py}}: Fornece a classe \texttt{BenchmarkRunner} para executar benchmarks em séries de grafos e recolher métricas de desempenho.

    \item \textbf{\texttt{src/graph\_generator.py}}: Implementa a classe \texttt{GraphGenerator} para gerar grafos aleatórios com vértices representados como pontos 2D e densidades de arestas configuráveis.

    \item \textbf{\texttt{src/visualizer.py}}: Contém classes para visualização de grafos (\texttt{GraphVisualizer}) e resultados experimentais (\texttt{ResultsVisualizer}).

    \item \textbf{\texttt{main.py}}: Interface de linha de comandos que expõe todas as funcionalidades através de comandos \texttt{typer}.
\end{itemize}

\subsection{Comandos Disponíveis}

O projecto expõe quatro comandos principais através da interface de linha de comandos:

\subsubsection{Geração de Grafos}

Gera grafos aleatórios para experimentação:

\begin{verbatim}
uv run python main.py generate [opções]
\end{verbatim}

\textbf{Opções:}
\begin{itemize}
    \item \texttt{--seed <número>}: Seed aleatória para reprodutibilidade (padrão: 112974)
    \item \texttt{--min-vertices <n>}: Número mínimo de vértices (padrão: 4)
    \item \texttt{--max-vertices <n>}: Número máximo de vértices (padrão: 12)
    \item \texttt{--densities <lista>}: Percentagens de densidade de arestas (padrão: 12.5 25.0 50.0 75.0)
    \item \texttt{--output-dir <directoria>}: Directoria de saída (padrão: experiments/graphs)
\end{itemize}

\textbf{Exemplo:}
\begin{verbatim}
uv run python main.py generate \
    --seed 112974 \
    --min-vertices 4 \
    --max-vertices 20 \
    --densities 12.5 25.0 50.0 75.0
\end{verbatim}

Este comando gera grafos com tamanhos de 4 a 20 vértices, cada um com quatro densidades diferentes, totalizando 68 grafos.

\subsubsection{Resolver um Grafo}

Resolve o problema Maximum Weight Clique para um grafo específico:

\begin{verbatim}
uv run python main.py solve <caminho_grafo> [opções]
\end{verbatim}

\textbf{Opções:}
\begin{itemize}
    \item \texttt{---visualize}: Gera visualização do grafo com o clique destacado
    \item \texttt{---no-show}: Guarda a visualização sem a mostrar
    \item \texttt{---mode <modo>}: Modo de execução: \texttt{both}, \texttt{exhaustive} ou \texttt{heuristic} (padrão: both)
\end{itemize}

\textbf{Exemplo:}
\begin{verbatim}
uv run python main.py solve \
    experiments/graphs/graph_n10_d50.graphml \
    --visualize \
    --mode both
\end{verbatim}

\subsubsection{Executar Benchmarks}

Executa benchmarks em séries de grafos e recolhe métricas de desempenho:

\begin{verbatim}
uv run python main.py benchmark [opções]
\end{verbatim}

\textbf{Opções:}
\begin{itemize}
    \item \texttt{---graphs-dir <directoria>}: Directoria com grafos (padrão: experiments/graphs)
    \item \texttt{---output-dir <directoria>}: Directoria de saída (padrão: experiments/results)
    \item \texttt{---exhaustive <intervalo>}: Intervalo de vértices para algoritmo exaustivo (ex: \texttt{4..15} ou \texttt{all})
    \item \texttt{---heuristic <intervalo>}: Intervalo de vértices para heurística gulosa (ex: \texttt{4..100} ou \texttt{all})
    \item \texttt{---plot}: Gera gráficos após o benchmarking
    \item \texttt{---verbose}: Mostra progresso detalhado (padrão: True)
\end{itemize}

\textbf{Exemplo com intervalos separados:}
\begin{verbatim}
uv run python main.py benchmark \
    --exhaustive 4..20 \
    --heuristic 4..500 \
    --plot
\end{verbatim}

Este comando executa o algoritmo exaustivo em grafos com 4 a 20 vértices e a heurística gulosa em grafos com 4 a 500 vértices, gerando gráficos no final.

\textbf{Exemplo simples:}
\begin{verbatim}
uv run python main.py benchmark --plot
\end{verbatim}

Executa ambos os algoritmos em todos os grafos disponíveis.

\subsubsection{Visualizar Resultados}

Gera gráficos a partir de resultados de benchmarks existentes:

\begin{verbatim}
uv run python main.py visualize [caminho_resultados] [opções]
\end{verbatim}

\textbf{Opções:}
\begin{itemize}
    \item \texttt{<caminho\_resultados>}: Caminho para ficheiro JSON de resultados (padrão: \texttt{experiments/results/benchmark\_results.json})
    \item \texttt{--output-dir <directoria>}: Directoria de saída para gráficos (padrão: experiments/plots)
\end{itemize}

\textbf{Exemplo:}
\begin{verbatim}
uv run python main.py visualize \
    experiments/results/benchmark_results.json \
    --output-dir experiments/plots
\end{verbatim}

Gera os seguintes gráficos:
\begin{itemize}
    \item \texttt{execution\_time.png}: Tempo de execução vs. número de vértices
    \item \texttt{operations\_count.png}: Número de operações vs. número de vértices
    \item \texttt{configurations\_tested.png}: Configurações testadas vs. número de vértices
    \item \texttt{heuristic\_precision.png}: Precisão da heurística vs. número de vértices
\end{itemize}

\subsection{Fluxo de Trabalho Recomendado}

Um fluxo de trabalho típico para experimentação:

\begin{enumerate}
    \item \textbf{Gerar grafos:}
          \begin{verbatim}
    uv run python main.py generate \
        --seed 112974 \
        --min-vertices 4 \
        --max-vertices 20
    \end{verbatim}

    \item \textbf{Executar benchmarks:}
          \begin{verbatim}
    uv run python main.py benchmark \
        --exhaustive 4..20 \
        --heuristic 4..500 \
        --plot
    \end{verbatim}

    \item \textbf{Visualizar um grafo específico:}
          \begin{verbatim}
    uv run python main.py solve \
        experiments/graphs/graph_n10_d50.graphml \
        --visualize
    \end{verbatim}

    \item \textbf{Gerar gráficos adicionais:}
          \begin{verbatim}
    uv run python main.py visualize
    \end{verbatim}
\end{enumerate}

\subsection{Formato dos Dados}

\subsubsection{Grafos (GraphML)}

Os grafos são guardados no formato GraphML, que preserva:
\begin{itemize}
    \item Coordenadas 2D dos vértices (\texttt{x}, \texttt{y})
    \item Pesos dos vértices (\texttt{weight})
    \item Estrutura de arestas
\end{itemize}

\subsubsection{Resultados de Benchmarks}

Os resultados são guardados em dois formatos:

\begin{itemize}
    \item \textbf{CSV}: Formato tabular para análise em folhas de cálculo
    \item \textbf{JSON}: Formato estruturado para processamento programático
\end{itemize}

Cada resultado contém:
\begin{itemize}
    \item Propriedades do grafo (número de vértices, arestas, densidade)
    \item Métricas do algoritmo exaustivo (clique, peso, tempo, operações, configurações)
    \item Métricas da heurística gulosa (clique, peso, tempo, operações, configurações)
    \item Métricas de comparação (precisão, factor de aceleração)
\end{itemize}

\subsection{Notas de Implementação}

\begin{itemize}
    \item O algoritmo exaustivo testa todos os $2^n$ subconjuntos possíveis de vértices, garantindo optimalidade mas com complexidade exponencial.

    \item A heurística gulosa utiliza uma estratégia multi-início, iniciando a construção de um clique a partir de cada vértice e seleccionando iterativamente o vértice compatível de maior peso.

    \item As métricas de operações básicas contam verificações de adjacência, sendo independentes do hardware e úteis para análise de complexidade.

    \item O gerador de grafos garante uma distância mínima entre vértices para evitar sobreposição espacial e utiliza seeds para reprodutibilidade.
\end{itemize}

\clearpage
\section*{Anexo B: Resultados Completos dos Benchmarks} \label{attach:complete_results}

Este anexo apresenta todos os resultados experimentais obtidos nos benchmarks realizados. A Tabela~\ref{tab:results_summary} na Secção~\ref{sec:testes} apresenta uma selecção representativa destes dados. A tabela completa abaixo contém todos os 68 resultados experimentais.

\begin{table}[!htbp]
\centering
\tiny
\caption{Resultados Completos dos Benchmarks. Legenda: V (Vértices), E (Arestas), $\rho$ (Densidade), Ex.(s) (Tempo Exaustivo), Gul.(s) (Tempo Guloso), Prec.(\%) (Precisão Heurística), Op.Ex. (Operações Exaustivo), Op.Gul. (Operações Guloso).}
\label{tab:complete_results}
\begin{tabular}{@{}ccccccccc@{}}
\toprule
V & E & $\rho$ & Ex.(s) & Gul.(s) & Prec.(\%) & Op.Ex. & Op.Gul. \\
\midrule
4 & 0 & 12.5 & 0.0000 & 0.0000 & 100.0 & 11 & 12 \\
4 & 1 & 12.5 & 0.0000 & 0.0000 & 100.0 & 11 & 16 \\
4 & 3 & 50.0 & 0.0000 & 0.0000 & 100.0 & 11 & 20 \\
4 & 4 & 75.0 & 0.0000 & 0.0000 & 100.0 & 17 & 24 \\
5 & 1 & 12.5 & 0.0000 & 0.0000 & 100.0 & 26 & 26 \\
5 & 2 & 25.0 & 0.0000 & 0.0000 & 100.0 & 34 & 32 \\
5 & 5 & 50.0 & 0.0000 & 0.0000 & 100.0 & 45 & 54 \\
5 & 7 & 75.0 & 0.0000 & 0.0000 & 100.0 & 47 & 62 \\
6 & 1 & 12.5 & 0.0001 & 0.0000 & 100.0 & 64 & 38 \\
6 & 3 & 25.0 & 0.0001 & 0.0000 & 100.0 & 76 & 49 \\
6 & 7 & 50.0 & 0.0001 & 0.0000 & 100.0 & 84 & 65 \\
6 & 11 & 75.0 & 0.0001 & 0.0000 & 100.0 & 118 & 118 \\
7 & 2 & 12.5 & 0.0001 & 0.0000 & 100.0 & 124 & 59 \\
7 & 5 & 25.0 & 0.0001 & 0.0000 & 100.0 & 133 & 79 \\
7 & 10 & 50.0 & 0.0001 & 0.0001 & 100.0 & 177 & 117 \\
7 & 15 & 75.0 & 0.0001 & 0.0001 & 100.0 & 285 & 196 \\
8 & 3 & 12.5 & 0.0002 & 0.0000 & 100.0 & 342 & 84 \\
8 & 7 & 25.0 & 0.0002 & 0.0000 & 100.0 & 262 & 109 \\
8 & 14 & 50.0 & 0.0002 & 0.0001 & 100.0 & 492 & 160 \\
8 & 21 & 75.0 & 0.0003 & 0.0001 & 100.0 & 1,096 & 358 \\
9 & 4 & 12.5 & 0.0004 & 0.0000 & 100.0 & 595 & 113 \\
9 & 9 & 25.0 & 0.0003 & 0.0001 & 100.0 & 576 & 165 \\
9 & 18 & 50.0 & 0.0004 & 0.0001 & 100.0 & 1,083 & 231 \\
9 & 27 & 75.0 & 0.0006 & 0.0001 & 100.0 & 1,054 & 390 \\
10 & 5 & 12.5 & 0.0008 & 0.0001 & 100.0 & 1,186 & 168 \\
10 & 11 & 25.0 & 0.0007 & 0.0001 & 100.0 & 1,644 & 215 \\
10 & 22 & 50.0 & 0.0008 & 0.0001 & 100.0 & 1,765 & 338 \\
10 & 33 & 75.0 & 0.0014 & 0.0002 & 100.0 & 3,485 & 524 \\
11 & 6 & 12.5 & 0.0018 & 0.0001 & 100.0 & 2,126 & 177 \\
11 & 13 & 25.0 & 0.0015 & 0.0001 & 100.0 & 3,484 & 227 \\
11 & 27 & 50.0 & 0.0016 & 0.0001 & 100.0 & 4,941 & 366 \\
11 & 41 & 75.0 & 0.0021 & 0.0002 & 100.0 & 8,148 & 878 \\
12 & 8 & 12.5 & 0.0037 & 0.0002 & 100.0 & 5,340 & 253 \\
12 & 16 & 25.0 & 0.0036 & 0.0001 & 100.0 & 4,621 & 294 \\
12 & 33 & 50.0 & 0.0031 & 0.0001 & 100.0 & 9,656 & 451 \\
12 & 49 & 75.0 & 0.0045 & 0.0002 & 100.0 & 20,610 & 1,120 \\
13 & 9 & 12.5 & 0.0061 & 0.0001 & 100.0 & 8,631 & 285 \\
13 & 19 & 25.0 & 0.0058 & 0.0001 & 100.0 & 11,110 & 366 \\
13 & 39 & 50.0 & 0.0075 & 0.0002 & 100.0 & 15,049 & 708 \\
13 & 58 & 75.0 & 0.0082 & 0.0003 & 100.0 & 27,293 & 1,466 \\
14 & 11 & 12.5 & 0.0119 & 0.0001 & 100.0 & 17,516 & 383 \\
14 & 22 & 25.0 & 0.0120 & 0.0001 & 100.0 & 22,958 & 510 \\
14 & 45 & 50.0 & 0.0146 & 0.0003 & 100.0 & 40,071 & 854 \\
14 & 68 & 75.0 & 0.0168 & 0.0004 & 100.0 & 60,495 & 2,154 \\
15 & 13 & 12.5 & 0.0246 & 0.0001 & 100.0 & 34,766 & 422 \\
15 & 26 & 25.0 & 0.0250 & 0.0002 & 100.0 & 41,240 & 576 \\
15 & 52 & 50.0 & 0.0260 & 0.0002 & 100.0 & 46,749 & 1,005 \\
15 & 78 & 75.0 & 0.0327 & 0.0004 & 100.0 & 125,843 & 2,116 \\
16 & 15 & 12.5 & 0.0486 & 0.0002 & 100.0 & 89,287 & 453 \\
16 & 30 & 25.0 & 0.0481 & 0.0002 & 100.0 & 74,632 & 619 \\
16 & 60 & 50.0 & 0.0523 & 0.0002 & 100.0 & 118,621 & 1,096 \\
16 & 90 & 75.0 & 0.0651 & 0.0004 & 100.0 & 247,730 & 1,990 \\
17 & 17 & 12.5 & 0.0969 & 0.0002 & 100.0 & 171,024 & 535 \\
17 & 34 & 25.0 & 0.0956 & 0.0002 & 100.0 & 145,922 & 720 \\
17 & 68 & 50.0 & 0.1036 & 0.0003 & 100.0 & 262,699 & 1,210 \\
17 & 102 & 75.0 & 0.1340 & 0.0005 & 100.0 & 569,794 & 3,153 \\
18 & 19 & 12.5 & 0.1960 & 0.0002 & 100.0 & 272,968 & 654 \\
18 & 38 & 25.0 & 0.2003 & 0.0002 & 100.0 & 396,173 & 917 \\
18 & 76 & 50.0 & 0.2156 & 0.0003 & 100.0 & 599,625 & 1,382 \\
18 & 114 & 75.0 & 0.2690 & 0.0005 & 100.0 & 1,170,908 & 2,960 \\
19 & 21 & 12.5 & 0.3751 & 0.0002 & 100.0 & 538,765 & 733 \\
19 & 42 & 25.0 & 0.4082 & 0.0002 & 100.0 & 712,021 & 908 \\
19 & 85 & 50.0 & 0.4501 & 0.0005 & 100.0 & 1,282,576 & 2,142 \\
19 & 128 & 75.0 & 0.5639 & 0.0010 & 100.0 & 2,783,894 & 5,261 \\
20 & 23 & 12.5 & 0.8072 & 0.0002 & 100.0 & 1,116,260 & 877 \\
20 & 47 & 25.0 & 0.8010 & 0.0003 & 81.2 & 1,613,817 & 957 \\
20 & 95 & 50.0 & 0.9002 & 0.0004 & 100.0 & 2,526,647 & 2,205 \\
20 & 142 & 75.0 & 0.9021 & 0.0009 & 100.0 & 2,915,493 & 5,220 \\

\bottomrule
\end{tabular}
\end{table}



\end{document}
