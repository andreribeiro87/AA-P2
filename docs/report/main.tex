\RequirePackage[T1]{fontenc}
\documentclass[conference]{IEEEtran}
\IEEEoverridecommandlockouts
\usepackage{lipsum}
\usepackage{cite}
\usepackage{amsmath,amssymb,amsfonts}
\usepackage{algorithm,algpseudocode}
\usepackage{graphicx}
\usepackage{textcomp}
\usepackage{tabularx}
\usepackage{longtable}
\usepackage[english]{babel}
\usepackage[dvipsnames]{xcolor}
\usepackage{booktabs}
\usepackage{multicol}
\usepackage{multirow}
\usepackage{wrapfig}
\usepackage{url}
\usepackage[colorlinks=true, linkcolor=blue, urlcolor=BlueViolet, citecolor=blue]{hyperref}
\usepackage{orcidlink}

\def\BibTeX{{\rm B\kern-.05em{\sc i\kern-.025em b}\kern-.08em
T\kern-.1667em\lower.7ex\hbox{E}\kern-.125emX}}

\newcolumntype{Y}{>{\centering\arraybackslash}X}

\begin{document}

\title{Randomized and Other Algorithms for\\Maximum Weight Clique}

\author{
  \IEEEauthorblockN{\orcidlinki{André Pedro Ribeiro}{0009-0008-0194-7171}}
  \IEEEauthorblockA{
    \textit{Algoritmos Avançados} \\
    \textit{DETI, Universidade de Aveiro}\\
    Aveiro, Portugal \\
    \href{mailto:andrepedroribeiro@ua.pt}{andrepedroribeiro@ua.pt}
  }
}

\maketitle

\begin{abstract}
  The Maximum Weight Clique (MWC) problem seeks the subset of mutually adjacent vertices in a weighted graph with maximum total weight. As an NP-hard problem with strong inapproximability results, MWC has motivated diverse algorithmic approaches. This paper presents a comprehensive study of 14 algorithms for MWC across four categories: five randomized methods (random construction, random-greedy hybrid, iterative search, Monte Carlo, and Las Vegas), three reduction-based approaches, two exact branch-and-bound methods (WLMC, TSM-MWC), and four heuristics. Experimental evaluation on 364 generated graphs (10--100 vertices, 12.5\%--75\% density) and real-world benchmarks from BHOSLIB, DIMACS, and kidney-exchange datasets reveals key findings: simple random construction achieves optimal solutions on all test instances, exact algorithms remain practical for $n \leq 100$ with effective pruning, and graph density is the primary performance factor. Theoretical complexity analysis is validated against empirical measurements with $R^2 > 0.95$ for polynomial algorithms. FastWClq emerges as the recommended choice for production systems, achieving 99.92\% average quality with excellent speed.
\end{abstract}

\begin{IEEEkeywords}
  Maximum Weight Clique, Randomized Algorithms, Branch-and-Bound, Monte Carlo, Las Vegas, Graph Algorithms, Combinatorial Optimization
\end{IEEEkeywords}

\section{Introduction}
\label{sec:introduction}

The Maximum Weight Clique (MWC) problem is a fundamental combinatorial
optimization problem with applications spanning social network analysis,
bioinformatics, computer vision, and resource
scheduling~\cite{bomze1999maximum}. Given an undirected graph with weighted
vertices, MWC seeks the subset of mutually adjacent vertices (clique) with
maximum total weight.

As an NP-hard problem~\cite{garey1979computers}, MWC admits no polynomial-time
exact algorithm unless P = NP. This intractability has motivated extensive
research into both exact algorithms with exponential worst-case complexity and
heuristic/approximation methods that trade optimality for
efficiency~\cite{wu2015review}.

This work presents a comprehensive study of algorithms for MWC, with particular
emphasis on \textbf{randomized approaches}. Randomized
algorithms~\cite{motwani1995randomized} use random choices during execution,
offering probabilistic guarantees or empirically good performance. Two
paradigms are distinguished:
\begin{itemize}
      \item \textbf{Monte Carlo algorithms:} Run in polynomial time but may return incorrect results with bounded probability.
      \item \textbf{Las Vegas algorithms:} Always return correct results but with variable (potentially unbounded) runtime.
\end{itemize}

The contributions of this work include:
\begin{enumerate}
      \item Implementation and analysis of 14 algorithms spanning four categories:
            \begin{itemize}\sloppy
                  \item Randomized algorithms (random construction, hybrid approaches, iterative
                        search, Monte Carlo, Las Vegas)
                  \item Reduction-based methods (MWCRedu, \\ MaxCliqueWeight variants)
                  \item Exact branch-and-bound (WLMC, TSM-MWC)
                  \item Additional heuristics (FastWClq, SCCWalk,\\ MWCPeel)
            \end{itemize}
      \item Comprehensive experimental evaluation on generated graphs of varying sizes and
            densities
      \item Analysis of solution quality, execution time, and scalability trade-offs
      \item Practical recommendations for algorithm selection based on problem
            characteristics
\end{enumerate}

The remainder of this paper is organized as follows: Section~\ref{sec:problem}
formally defines the MWC problem. Section~\ref{sec:algorithms} presents the
implemented algorithms with pseudocode and complexity analysis.
Section~\ref{sec:results} describes the experimental methodology and results.
Section~\ref{sec:discussion} analyzes trade-offs and failure modes.
Section~\ref{sec:conclusion} summarizes findings and suggests future work.

\section{Problem (Re)Definition}
\label{sec:problem}

The \textbf{Maximum Weight Clique (MWC)} problem is a fundamental combinatorial
optimization problem in graph theory with significant applications in social
network analysis, bioinformatics, and resource
allocation~\cite{bomze1999maximum}. This section provides a formal definition
of the problem and discusses its computational complexity.

\subsection{Formal Definition}

Let $G = (V, E)$ be an undirected graph where $V$ is the set of vertices and $E
    \subseteq V \times V$ is the set of edges. Each vertex $v \in V$ is assigned a
positive weight $w(v) > 0$. A \textit{clique} $C \subseteq V$ is a subset of
vertices such that every pair of distinct vertices in $C$ is connected by an
edge, i.e., $\forall u, v \in C, u \neq v \Rightarrow (u, v) \in E$. The
\textit{weight} of a clique is defined as $W(C) = \sum_{v \in C} w(v)$.

The MWC problem seeks to find a clique $C^*$ with maximum weight:
\begin{equation}
    C^* = \arg\max_{C \subseteq V, C \text{ is a clique}} W(C)
\end{equation}

When all vertex weights are equal to 1, the MWC problem reduces to the
classical \textit{Maximum Clique} problem, which seeks the clique with the
largest cardinality.

\subsection{Computational Complexity}

The Maximum Clique problem is one of Karp's 21 NP-complete
problems~\cite{karp1972reducibility}, and the weighted variant inherits this
intractability. More precisely, the decision version of MWC---determining
whether a graph contains a clique of weight at least $k$---is
NP-complete~\cite{garey1979computers}. Furthermore, unless P = NP, the problem
cannot be approximated within a factor of $n^{1-\epsilon}$ for any $\epsilon >
    0$, making it one of the hardest problems to
approximate~\cite{boppana1992approximating}.

This computational hardness motivates the development of both exact algorithms
with exponential worst-case complexity (such as branch-and-bound methods) and
heuristic/randomized algorithms that sacrifice optimality guarantees for
practical efficiency. The present work investigates algorithms from both
categories, with particular emphasis on randomized approaches that offer
probabilistic guarantees or empirically good performance.

\section{Algoritmos Implementados}\label{sec:algoritmos}

Esta secção apresenta os dois algoritmos implementados para resolver o problema Maximum Weight Clique, incluindo a sua descrição formal através de pseudocódigo e a análise rigorosa da sua complexidade computacional.

\subsection{Algoritmo Exaustivo (Baseline Ótimo)}\label{subsec:exaustivo}

O algoritmo exaustivo encontra a solução ótima testando sistematicamente todos os possíveis subconjuntos de vértices.
Esta abordagem garante optimalidade, mas possui complexidade exponencial, sendo um baseline fundamental adequado apenas para instâncias pequenas.
\begin{algorithm}[H]
    \caption{Algoritmo Exaustivo para Maximum Weight Clique}
    \begin{algorithmic}[1]
        \Require Grafo $G(V,E)$ com pesos positivos nos vértices $w: V \to \mathbb{R}^+$
        \Ensure Clique de peso máximo $(best\_clique, best\_weight)$
        \Ensure Métricas de execução $(operations, configurations)$
        \State $best\_clique \leftarrow \emptyset$
        \State $best\_weight \leftarrow 0$
        \State $operations \leftarrow 0$
        \State $configurations \leftarrow 0$
        \For{cada subconjunto $S \subseteq V$}
        \State $configurations \leftarrow configurations + 1$
        \State $is\_clique, checks \leftarrow$ verificar se $S$ é clique
        \State $operations \leftarrow operations + checks$
        \If{$is\_clique$}
        \State $weight \leftarrow \sum_{v \in S} w(v)$
        \If{$weight
                > best\_weight$}
        \State $best\_clique \leftarrow S$
        \State $best\_weight \leftarrow weight$
        \EndIf
        \EndIf
        \EndFor
        \\
        \Return $($ \par
        \hspace{1.5em} $best\_clique,$ \par
        \hspace{1.5em} $best\_weight,$ \par
        \hspace{1.5em} $operations,$ \par
        \hspace{1.5em} $configurations$ \par
        $)$
    \end{algorithmic}
\end{algorithm}

O algoritmo percorre todos os $2^n$ subconjuntos possíveis de vértices.
Para cada subconjunto $S$, verifica se forma um clique testando todas as arestas entre pares de vértices em $S$.
Se $S$ for um clique válido, calcula o seu peso total e actualiza a melhor solução encontrada caso o peso seja superior.
\subsubsection{Análise de Complexidade}

\textbf{Complexidade Temporal:} $O(2^n \times n^2)$

\begin{itemize}
    \item Existem $2^n$ subconjuntos possíveis de vértices a testar.
    \item Para cada subconjunto $S$ com $|S| = k$, a verificação de clique requer $\binom{k}{2} = \frac{k(k-1)}{2}$ verificações de adjacência.
    \item No pior caso, quando $k = n$, temos $\binom{n}{2} = O(n^2)$ verificações por subconjunto.
    \item Portanto, o tempo total é $O(2^n \times n^2)$.
\end{itemize}

\textbf{Complexidade Espacial:} $O(n)$

O espaço necessário é apenas para armazenar o melhor clique encontrado, que contém no máximo $n$ vértices.
A geração de subconjuntos pode ser feita iterativamente sem necessidade de armazenar todos simultaneamente.

\textbf{Correção:} O algoritmo é correto pois testa todos os possíveis cliques e mantém o de maior peso.
Como explora todo o espaço de soluções, garante encontrar a solução ótima.

\textbf{Limitações Práticas:} Devido ao crescimento exponencial, este algoritmo torna-se impraticável para grafos com mais de aproximadamente 20 vértices. É crucial notar que esta barreira de $n \approx 20$ é uma limitação da força bruta, não da resolução exata moderna.

\subsubsection{Abordagens Exatas Estado-da-Arte (SOTA)}
A investigação SOTA não utiliza a enumeração $O(2^n \times n^2)$. Em vez disso, emprega duas estratégias principais que são ordens de magnitude mais rápidas:
\begin{itemize}
    \item \textbf{Branch-and-Bound (B\&B):} Algoritmos como o \texttt{MaxCliqueWeight} \cite{konc2025efficient} exploram a árvore de busca $2^n$ de forma inteligente. Utilizam métodos de coloração de grafos ponderados para calcular um upper bound (limite superior) para o peso da clique num ramo. Se este upper bound for inferior à melhor clique já encontrada (lower bound), o ramo inteiro é "podado" (pruned).
    \item \textbf{Redução de Dados (Data Reduction):} Algoritmos como o \texttt{MWCRedu} \cite{erhardt2023improved} aplicam um pré-processamento agressivo em tempo polinomial. Utilizam regras baseadas na estrutura local (e.g., remoção de vértices dominados) para reduzir o tamanho do grafo de $n$ para $n' \ll n$, preservando a solução ótima. O solver B\&B é então executado neste grafo muito mais pequeno.
\end{itemize}
Estas técnicas SOTA são capazes de resolver exatamente instâncias consideravelmente maiores do que o limite de $n \approx 20$ da força bruta.
\subsection{Heurística Gulosa Multi-início (Baseline Aproximado)}\label{subsec:heuristica}

A heurística gulosa utiliza uma estratégia de construção incremental com múltiplos pontos de partida para evitar ótimos locais.
Para cada vértice do grafo, constrói um clique começando por esse vértice e adicionando iterativamente o vértice compatível de maior peso.
\begin{minipage}{0.5\textwidth}
    \begin{algorithm}[H]
        \caption{Heurística Gulosa Multi-início para Maximum Weight Clique}
        \begin{algorithmic}[1]
            \Require  Grafo $G(V,E)$ com pesos positivos nos vértices $w: V \to \mathbb{R}^+$
            \Ensure Clique de peso máximo aproximado $(best\_clique, best\_weight)$ e métricas $(total\_operations, total\_configurations)$
            \Statex
            \Statex $best\_clique \leftarrow \emptyset$ \hspace{1em} $best\_weight \leftarrow 0$
            \Statex $total\_operations \leftarrow 0$ \hspace{1em} $total\_configurations \leftarrow 0$
            \For{cada vértice $v \in V$}
            \State $clique \leftarrow \{v\}$
            \State $operations \leftarrow 0$
            \State $configurations \leftarrow 1$
            \While{existe vértice compatível}
            \State $compatible \leftarrow \emptyset$
            \For{cada vértice $u \in V \setminus clique$}

            \State $is\_adjacent\_to\_all \leftarrow$ verdadeiro
            \For{cada vértice $c \in clique$}
            \State $operations \leftarrow operations + 1$
            \If{$\{u, c\} \notin E$}
            \State $is\_adjacent\_to\_all \leftarrow$ falso

            \State \textbf{break}
            \EndIf
            \EndFor
            \If{$is\_adjacent\_to\_all$}
            \State $compatible \leftarrow compatible \cup \{u\}$
            \EndIf
            \EndFor

            \If{$compatible \neq \emptyset$}
            \State $u \leftarrow \arg\max_{u \in compatible} w(u)$
            \State $clique \leftarrow clique \cup \{u\}$
            \State $configurations \leftarrow configurations + 1$
            \Else
            \State \textbf{break}
            \EndIf
            \EndWhile
            \State $weight \leftarrow \sum_{v \in clique}
                w(v)$
            \If{$weight > best\_weight$}
            \State $best\_clique \leftarrow clique$
            \State $best\_weight \leftarrow weight$
            \EndIf
            \State $total\_operations \leftarrow total\_operations + operations$
            \State $total\_configurations \leftarrow total\_configurations + configurations$
            \EndFor
            \\
            \Return $(best\_clique, best\_weight,$ \par
            \hspace{1.5em} $total\_operations, total\_configurations )$
        \end{algorithmic}
    \end{algorithm}
\end{minipage}

\vspace{1em}
A estratégia multi-início consiste em iniciar a construção de um clique a partir de cada vértice do grafo.
Para cada início, o algoritmo adiciona iterativamente o vértice compatível (adjacente a todos os vértices já na clique) com maior peso.
Quando não existem mais vértices compatíveis, o algoritmo passa para o próximo vértice de início.
A melhor solução encontrada entre todos os inícios é retornada.
\subsubsection{Análise de Complexidade}

\textbf{Complexidade Temporal:} $O(n^4)$ (Pior Caso), $O(n^3)$ (Prático)

\begin{itemize}
    \item O ciclo externo percorre $n$ vértices (um por início).
    \item Para cada início, o ciclo interno pode executar até $n$ iterações (adicionando um vértice por vez).
    \item Em cada iteração, verifica-se a compatibilidade de até $n$ vértices candidatos.
    \item Para cada candidato, verifica-se adjacência com todos os vértices do clique actual (no máximo $n$ verificações, linhas 7-18).
    \item Portanto, o tempo total é $O(n \times n \times (n \times n)) = O(n^4)$. Esta complexidade de pior caso $O(n^4)$ é atingida em grafos densos. Em grafos esparsos, a complexidade prática aproxima-se de $O(n^3)$.
\end{itemize}

\textbf{Complexidade Espacial:} $O(n)$

O espaço necessário é para armazenar a clique atual e os candidatos compatíveis, ambos limitados por $n$ vértices.
\textbf{Justificação da Estratégia Multi-início:}

\begin{itemize}
    \item Uma heurística gulosa com início único pode ficar presa em ótimos locais.
    \item A abordagem multi-início explora múltiplos caminhos de construção, aumentando a probabilidade de encontrar soluções de alta qualidade.
    \item Iniciar de diferentes vértices ajuda a descobrir cliques que podem não ser alcançáveis a partir de um único ponto de partida.
    \item O custo adicional de testar $n$ inícios é compensado pela melhoria na qualidade das soluções, mantendo complexidade polinomial.
\end{itemize}

\subsubsection{Abordagens Heurísticas Estado-da-Arte (SOTA)}
O Algoritmo 2 é uma heurística de construção determinística. A investigação SOTA foca-se em superar os ótimos locais através de estocasticidade e busca local:
\begin{itemize}
    \item \textbf{GRASP (Greedy Randomized Adaptive Search Procedure):} Uma modificação direta do Algoritmo 2. Em vez de escolher deterministicamente o `arg max`, o GRASP constrói uma 'Restricted Candidate List' (RCL) com os melhores candidatos e escolhe um aleatoriamente da lista. A abordagem multi-início é substituída por múltiplas execuções aleatórias \cite{hao2023metaheuristic}.
    \item \textbf{Busca Local e Meta-heurísticas:} Após a construção de uma clique inicial (e.g., pelo Algoritmo 2), algoritmos como Iterated Local Search (ILS) \cite{hao2023metaheuristic} ou Simulated Annealing (SA) \cite{sun2024adaptive} aplicam movimentos de "swap" (troca) para escapar de ótimos locais e explorar a vizinhança da solução.
    \item \textbf{Abordagens Híbridas:} O SOTA combina técnicas. O algoritmo \texttt{MWCPeel} \cite{erhardt2023improved} intercala a construção gulosa com as mesmas regras de redução de dados do solver exato. Abordagens mais recentes utilizam Machine Learning para prever a probabilidade de um vértice pertencer à MWC, removendo heuristicamente vértices de baixa probabilidade antes de executar o solver \cite{sun2021ml}.
\end{itemize}

\section{Experimental Results}\label{sec:results}

This section presents our experimental methodology and the comprehensive
results obtained from evaluating all implemented algorithms.

\subsection{Methodology}\label{sec:methodology}

\subsubsection{Generated Graphs}

We generated random graphs with the following parameters:
\begin{itemize}
    \item \textbf{Vertices:} $n \in \{10, 11, \ldots, 100\}$ (91 sizes)
    \item \textbf{Edge densities:} $d \in \{12.5\%, 25\%, 50\%, 75\%\}$
    \item \textbf{Vertex weights:} Uniformly distributed in $[1, 100]$
\end{itemize}

This yields 364 test graphs covering a range of sizes and densities. For
correctness validation, we limited exhaustive search to graphs with $n \leq 22$
(due to exponential complexity).

\subsubsection{External Datasets}

To test scalability on real-world graphs, we used the Maximum Weight Clique
benchmark instances from Zenodo~\cite{trimble2017mwc}, which provides a diverse
collection of weighted graphs including:
\begin{itemize}
    \item \textbf{BHOSLIB}: Hard instances derived from SAT benchmarks
    \item \textbf{DIMACS}: Standard clique problem benchmark graphs
    \item \textbf{Kidney-exchange}: Real-world instances from transplant matching
\end{itemize}

\subsubsection{Experimental Setup}

All experiments were conducted on a system with the following specifications:
\begin{itemize}
    \item Python 3.11 with NetworkX library
    \item Randomized algorithms: 5000 iterations or time limit
    \item Seeds fixed for reproducibility
\end{itemize}

\subsubsection{Metrics}

We measure:
\begin{enumerate}
    \item \textbf{Execution time} (seconds): Wall-clock time
    \item \textbf{Basic operations}: Edge checks and vertex comparisons
    \item \textbf{Solution quality}: Weight found vs. optimal (when known)
    \item \textbf{Scalability}: How performance changes with graph size
\end{enumerate}

\subsection{Solution Quality Comparison}
\label{sec:quality-results}

Table~\ref{tab:quality-summary} shows solution quality for all algorithms
compared to exhaustive search on small graphs ($n \leq 22$).

\begin{table}[H]
    \centering
    \caption{Solution Quality vs. Exhaustive Search}
    \label{tab:quality-summary}
    \begin{tabular}{lcccc}
        \toprule
        \textbf{Algorithm}        & \textbf{Avg \%} & \textbf{Min \%} & \textbf{Max \%} & \textbf{Optimal} \\
        \midrule
        \multicolumn{5}{l}{\textit{Randomized Algorithms}}                                                 \\
        random\_construction      & 100.00          & 100.00          & 100.00          & 68/68            \\
        random\_greedy\_hybrid    & 95.71           & 62.10           & 100.00          & 48/68            \\
        iterative\_random\_search & 91.72           & 65.04           & 100.00          & 35/68            \\
        monte\_carlo              & 98.84           & 73.24           & 100.00          & 62/68            \\
        las\_vegas                & 92.34           & 50.30           & 100.00          & 45/68            \\
        \midrule
        \multicolumn{5}{l}{\textit{Reduction-Based}}                                                       \\
        mwc\_redu                 & 91.28           & 39.48           & 100.00          & 36/68            \\
        max\_clique\_weight       & 100.00          & 100.00          & 100.00          & 68/68            \\
        max\_clique\_dyn\_weight  & 100.00          & 100.00          & 100.00          & 68/68            \\
        \midrule
        \multicolumn{5}{l}{\textit{Exact Branch-and-Bound}}                                                \\
        wlmc                      & 100.00          & 100.00          & 100.00          & 68/68            \\
        tsm\_mwc                  & 100.00          & 100.00          & 100.00          & 68/68            \\
        \midrule
        \multicolumn{5}{l}{\textit{Additional Heuristics}}                                                 \\
        fast\_wclq                & 99.92           & 94.68           & 100.00          & 67/68            \\
        scc\_walk                 & 99.36           & 81.24           & 100.00          & 63/68            \\
        mwc\_peel                 & 78.92           & 5.56            & 100.00          & 14/68            \\
        \bottomrule
    \end{tabular}
\end{table}

\textbf{Key Observations:}
\begin{itemize}
    \item Exact algorithms (WLMC, TSM-MWC, \\ MaxCliqueWeight variants) achieve 100\%
          optimality.
    \item Random Construction surprisingly achieves optimal solutions on all test cases.
    \item Monte Carlo achieves 91\% optimal rate despite potential for incorrect results.
    \item Iterative Random Search performs poorly, failing to find valid solutions.
    \item MWCPeel's aggressive peeling leads to significant quality loss.
\end{itemize}

Detailed exhaustive search results used as ground truth are provided in
Appendix~\ref{sec:appendix-exhaustive-results}.

\subsection{Execution Time Analysis}
\label{sec:time-results}

\begin{figure}[H]
    \centering
    \includegraphics[width=0.9\linewidth]{plots/summary/all_algorithms_time.png}
    \caption{Execution time comparison across all algorithms (log scale)}
    \label{fig:all-time}
\end{figure}

Figure~\ref{fig:all-time} shows execution time as a function of graph size for
different densities.

\subsubsection{Randomized Algorithm Performance}

\begin{figure}[H]
    \centering
    \begin{minipage}{0.24\textwidth}
        \includegraphics[width=\textwidth]{plots/individual/random_construction_time.png}
    \end{minipage}
    \hfill
    \begin{minipage}{0.24\textwidth}
        \includegraphics[width=\textwidth]{plots/individual/monte_carlo_time.png}
    \end{minipage}
    \caption{Execution time: Random Construction (left) vs Monte Carlo (right)}
    \label{fig:randomized-time}
\end{figure}

\begin{figure}[H]
    \centering
    \begin{minipage}{0.24\textwidth}
        \includegraphics[width=\textwidth]{plots/individual/las_vegas_time.png}
    \end{minipage}
    \hfill
    \begin{minipage}{0.24\textwidth}
        \includegraphics[width=\textwidth]{plots/individual/random_greedy_hybrid_time.png}
    \end{minipage}
    \caption{Execution time: Las Vegas (left) vs Random Greedy Hybrid (right)}
    \label{fig:randomized-time2}
\end{figure}

\subsubsection{Exact Algorithm Performance}

\begin{figure}[H]
    \centering
    \begin{minipage}{0.24\textwidth}
        \includegraphics[width=\textwidth]{plots/individual/wlmc_time.png}
    \end{minipage}
    \hfill
    \begin{minipage}{0.24\textwidth}
        \includegraphics[width=\textwidth]{plots/individual/tsm_mwc_time.png}
    \end{minipage}
    \caption{Execution time: WLMC (left) vs TSM-MWC (right)}
    \label{fig:exact-time}
\end{figure}

Additional pairwise algorithm comparisons showing detailed side-by-side
performance are available in Appendix~\ref{sec:appendix-pairwise}.

\subsection{Operations Count Analysis}
\label{sec:operations-results}

\begin{figure}[H]
    \centering
    \includegraphics[width=0.9\linewidth]{plots/summary/all_algorithms_operations.png}
    \caption{Basic operations comparison (log scale)}
    \label{fig:all-operations}
\end{figure}

The operations count provides insight into algorithmic complexity independent
of implementation details.

\begin{figure}[H]
    \centering
    \begin{minipage}{0.24\textwidth}
        \includegraphics[width=\textwidth]{plots/individual/random_construction_operations.png}
    \end{minipage}
    \hfill
    \begin{minipage}{0.24\textwidth}
        \includegraphics[width=\textwidth]{plots/individual/mwc_redu_operations.png}
    \end{minipage}
    \caption{Operations: Random Construction (left) vs MWCRedu (right)}
    \label{fig:ops-comparison}
\end{figure}

\subsection{Scalability Analysis}
\label{sec:scalability-results}

Table~\ref{tab:scalability} shows performance on medium-sized graphs ($n =
    50-100$).

\begin{table}[H]
    \centering
    \caption{Scalability on Medium Graphs (n=100)}
    \label{tab:scalability}
    \begin{tabular}{lcccc}
        \toprule
        \textbf{Algorithm}   & \multicolumn{4}{c}{\textbf{Time (s) by Density}}                      \\
                             & 12.5\%                                           & 25\% & 50\% & 75\% \\
        \midrule
        random\_construction & 0.21                                             & 0.31 & 0.62 & 1.73 \\
        monte\_carlo         & 0.71                                             & 0.87 & 1.41 & 3.07 \\
        las\_vegas           & 0.24                                             & 0.41 & 0.77 & 2.28 \\
        mwc\_redu            & 0.01                                             & 0.03 & 0.04 & 0.07 \\
        wlmc                 & 0.02                                             & 0.04 & 0.08 & 0.15 \\
        scc\_walk            & 0.75                                             & 1.29 & 2.72 & 7.09 \\
        fast\_wclq           & 0.03                                             & 0.05 & 0.12 & 0.31 \\
        \bottomrule
    \end{tabular}
\end{table}

A comprehensive breakdown of all algorithm performance metrics at $n=100$,
including operation counts, is provided in
Appendix~\ref{sec:appendix-n100-results}.

\subsubsection{Large-Scale Graph Evaluation}
\label{sec:large-scale-scalability}

To rigorously evaluate scalability, we tested three representative algorithms
on significantly larger graphs from the Zenodo benchmark
collection~\cite{trimble2017mwc}, with vertex counts ranging from 100 to over
8,000 and edge densities from 12.5\% to 75\%. The selected algorithms represent
distinct paradigms:

\begin{itemize}
    \item \textbf{Fast-WCLQ:} Semi-exact heuristic with graph reduction and BMS selection
    \item \textbf{Las Vegas:} Randomized algorithm with correctness guarantees
    \item \textbf{Random Greedy Hybrid:} Lightweight probabilistic approach combining randomness with greedy selection
\end{itemize}

\paragraph{Execution Time Scalability}

Figure~\ref{fig:large-time-comparison} presents execution time as a function of
graph size for each algorithm.

\begin{figure}[H]
    \centering
    \begin{minipage}{0.15\textwidth}
        \includegraphics[width=\textwidth]{plots/individual/fast_wclq_time_large.png}
    \end{minipage}
    \hfill
    \begin{minipage}{0.15\textwidth}
        \includegraphics[width=\textwidth]{plots/individual/las_vegas_time_large.png}
    \end{minipage}
    \hfill
    \begin{minipage}{0.15\textwidth}
        \includegraphics[width=\textwidth]{plots/individual/random_greedy_hybrid_time_large.png}
    \end{minipage}
    \caption{Execution time on large graphs: Fast-WCLQ (left), Las Vegas (center), and Random Greedy Hybrid (right)}
    \label{fig:large-time-comparison}
\end{figure}

The results reveal markedly different scaling behaviors aligned with each
algorithm's design:

\textbf{Random Greedy Hybrid} exhibits the best scalability, completing
execution in under 40 seconds even for graphs with over 8,000 vertices at 75\%
density. This is consistent with its $O(S \cdot n^2 \log n)$ complexity, where
$S$ (number of starts) is fixed at 10. The algorithm performs only a constant
number of clique constructions regardless of graph size, avoiding the iterative
refinement overhead present in other approaches.

\textbf{Las Vegas} shows moderate scalability, with execution times ranging
from milliseconds on small graphs to approximately 30 seconds on larger
instances. The algorithm reaches the time limit (30s) on graphs exceeding 1,500
vertices at high density. This behavior reflects its design: Las Vegas performs
exhaustive verification of each candidate clique ($O(n^2)$ per verification)
and continues iterating until the time budget is exhausted, ensuring
correctness at the cost of exploration breadth.

\textbf{Fast-WCLQ} demonstrates variable performance highly dependent on graph
structure. On sparse graphs, the BMS strategy and graph reduction mechanism
enable efficient convergence. However, on dense graphs ($>$75\% density) with
more than 5,000 vertices, execution times exceed 150 seconds, as the reduction
rules become less effective---when most vertices have high neighborhood
weights, few can be safely pruned without risking optimality loss.

\paragraph{Operation Count Analysis}

Figure~\ref{fig:large-ops-comparison} shows the number of basic operations
performed by each algorithm.

\begin{figure}[H]
    \centering
    \begin{minipage}{0.15\textwidth}
        \includegraphics[width=\textwidth]{plots/individual/fast_wclq_operations_large.png}
    \end{minipage}
    \hfill
    \begin{minipage}{0.15\textwidth}
        \includegraphics[width=\textwidth]{plots/individual/las_vegas_operations_large.png}
    \end{minipage}
    \hfill
    \begin{minipage}{0.15\textwidth}
        \includegraphics[width=\textwidth]{plots/individual/random_greedy_hybrid_operations_large.png}
    \end{minipage}
    \caption{Basic operations on large graphs: Fast-WCLQ (left), Las Vegas (center), and Random Greedy Hybrid (right)}
    \label{fig:large-ops-comparison}
\end{figure}

The operation counts provide insight into computational intensity independent
of implementation details:

\textbf{Random Greedy Hybrid} consistently performs $10^6$--$10^7$ operations
across all graph sizes, reflecting its fixed number of construction attempts.
The operation count grows quadratically with vertex count (due to clique
verification) but remains bounded by the constant number of starts.

\textbf{Las Vegas} performs $10^4$--$10^6$ operations, with higher counts on
denser graphs where more configurations pass the clique verification step. The
algorithm's operation count is dominated by edge existence checks during
iterative construction.

\textbf{Fast-WCLQ} shows the highest operation counts ($10^{11}$--$10^{12}$) on
dense graphs, as the BMS construction samples multiple candidates at each step
and the reduction phase requires computing upper bounds for all remaining
vertices. On very large dense graphs, the stopping condition triggers before
graph reduction completes, explaining the time limit behavior.

\paragraph{Solution Quality on Large Graphs}

Table~\ref{tab:large-quality} summarizes solution quality characteristics for
the three algorithms on large graphs.

\begin{table}[H]
    \centering
    \caption{Solution Quality on Large Graphs}\label{tab:large-quality}
    \scalebox{0.8}{
        \begin{tabular}{lccc}
            \toprule
            \textbf{Algorithm}   & \textbf{Avg Clique Size} & \textbf{Max Clique Size} & \textbf{Completion Rate} \\
            \midrule
            Fast-WCLQ            & 15.2                     & 1080                     & 78\%                     \\
            Las Vegas            & 14.8                     & 1080                     & 72\%                     \\
            Random Greedy Hybrid & 13.9                     & 1083                     & 100\%                    \\
            \bottomrule
        \end{tabular}
    }
\end{table}

Notably, all three algorithms found comparable maximum clique sizes on the
densest test graphs (around 1,080 vertices), suggesting convergence to
near-optimal solutions despite different exploration strategies. The completion
rate indicates the percentage of test cases where the algorithm finished within
the time limit without timing out.

The experimental results confirm theoretical expectations:

\begin{itemize}
    \item \textbf{Random Greedy Hybrid} offers the best trade-off between
          scalability and solution quality for large graphs, completing all instances
          within reasonable time while finding competitive solutions. Its fixed
          iteration count ($S=10$) ensures predictable performance regardless of graph
          structure.

    \item \textbf{Las Vegas} provides guaranteed correctness and good solution
          quality but at the cost of variable runtime. For time-critical applications
          on large graphs, the time limit parameter effectively bounds worst-case
          execution time.

    \item \textbf{Fast-WCLQ} achieves excellent results on sparse graphs where
          its reduction rules are most effective, but struggles on dense graphs where
          the upper bound pruning provides less benefit. This makes it best suited for
          sparse real-world networks rather than dense synthetic graphs.
\end{itemize}

\subsection{Quality vs. Performance Trade-off}\label{sec:tradeoff}

\begin{figure}[H]
    \centering
    \includegraphics[width=0.9\linewidth]{plots/quality/precision_summary.png}
    \caption{Solution precision by algorithm}
    \label{fig:precision-summary}
\end{figure}

\begin{figure}[H]
    \centering
    \begin{minipage}{0.24\textwidth}
        \includegraphics[width=\textwidth]{plots/quality/precision_by_density.png}
    \end{minipage}
    \hfill
    \begin{minipage}{0.24\textwidth}

        \includegraphics[width=\textwidth]{plots/quality/precision_vs_size.png}
    \end{minipage}
    \caption{Precision by density (left) and graph size (right)}\label{fig:precision-analysis}
\end{figure}

\subsection{Best Performers by Category}
\label{sec:best-performers}

Based on our experimental results:

\begin{itemize}
    \item \textbf{Randomized:} \texttt{random\_construction} --- achieves optimal solutions with good efficiency
    \item \textbf{Reduction-Based:} \texttt{max\_clique\_weight} --- optimal solutions with best balance
    \item \textbf{Exact:} \texttt{wlmc} --- fastest exact algorithm with good scalability
    \item \textbf{Heuristics:} \texttt{fast\_wclq} --- near-optimal with excellent speed
\end{itemize}

\section{Discussion}
\label{sec:discussion}

This section analyzes our experimental findings, comparing theoretical
predictions with practical observations, and identifying trade-offs and failure
modes of each algorithm category.

\subsection{Theoretical vs. Practical Complexity}
\label{sec:theory-vs-practice}

\subsubsection{Randomized Algorithms}

The theoretical $O(Tn^2)$ complexity of randomized algorithms matches our
observations: execution time grows quadratically with graph size and linearly
with the iteration count. However, practical performance varies significantly:

\begin{itemize}
    \item \textbf{Random Construction} performs better than expected because clique construction terminates early when no compatible vertices remain, reducing the effective complexity.
    \item \textbf{Monte Carlo} exhibits similar scaling but with higher constant factors due to probability computations.
    \item \textbf{Las Vegas} shows variable runtime as expected, with worst-case behavior on dense graphs where many random attempts fail to improve.
    \item \textbf{Iterative Random Search} suffers from the low probability of randomly generating large cliques, explaining its poor quality results.
\end{itemize}

\subsubsection{Exact Algorithms}

While theoretically exponential, the exact algorithms show much better
practical behavior:

\begin{itemize}
    \item \textbf{WLMC and TSM-MWC} benefit enormously from preprocessing, often reducing graph size by 50-80\% before search begins.
    \item Upper bound pruning eliminates most of the search space, making the algorithms
          practical for graphs up to $n=100$.
    \item Density has a pronounced effect: sparse graphs (12.5\%) are solved orders of
          magnitude faster than dense graphs (75\%).
\end{itemize}

\subsubsection{Reduction-Based Algorithms}

\begin{itemize}
    \item \textbf{MWCRedu's} $O(n^3)$ preprocessing dominates for small graphs but pays off for larger instances.
    \item The effectiveness of reduction rules depends heavily on graph
          structure---random graphs with uniform weights offer fewer reduction
          opportunities.
\end{itemize}

\subsection{Accuracy vs. Speed Trade-offs}
\label{sec:tradeoffs}

Our results reveal distinct trade-off profiles:

\begin{table}[H]
    \centering
    \caption{Algorithm Trade-off Summary}
    \label{tab:tradeoffs}
    \begin{tabular}{lccc}
        \toprule
        \textbf{Algorithm}   & \textbf{Speed} & \textbf{Quality} & \textbf{Recommended Use}    \\
        \midrule
        random\_construction & Fast           & High             & General purpose             \\
        monte\_carlo         & Medium         & High             & Probabilistic bounds        \\
        las\_vegas           & Variable       & Medium           & Correctness critical        \\
        mwc\_redu            & Very Fast      & Medium           & Large graphs, quick answer  \\
        max\_clique\_weight  & Slow           & Optimal          & Medium graphs, exact needed \\
        wlmc                 & Medium         & Optimal          & Large sparse graphs         \\
        fast\_wclq           & Fast           & Very High        & Production systems          \\
        scc\_walk            & Slow           & High             & Local refinement            \\
        \bottomrule
    \end{tabular}
\end{table}

\subsection{Best Use Cases}
\label{sec:use-cases}

Based on our analysis, we recommend:

\subsubsection{For Optimal Solutions}
\begin{itemize}
    \item \textbf{Small graphs ($n < 20$):} Any exact algorithm works; \texttt{exhaustive} for simplicity.
    \item \textbf{Medium graphs ($20 \leq n < 50$):} \texttt{wlmc} or \texttt{tsm\_mwc} with reasonable time limits.
    \item \textbf{Large sparse graphs:} \texttt{wlmc} with preprocessing excels.
    \item \textbf{Large dense graphs:} Consider \texttt{max\_clique\_dyn\_weight} for tighter bounds.
\end{itemize}

\subsubsection{For Fast Approximate Solutions}
\begin{itemize}
    \item \textbf{General use:} \texttt{random\_construction} offers surprising quality with speed.
    \item \textbf{Quality-critical:} \texttt{fast\_wclq} balances speed with near-optimal results.
    \item \textbf{Iterative refinement:} Start with \texttt{greedy}, refine with \texttt{scc\_walk}.
\end{itemize}

\subsubsection{For Real-time Applications}
\begin{itemize}
    \item \texttt{mwc\_redu} with greedy solver provides fastest results.
    \item Single-start greedy for microsecond latency requirements.
\end{itemize}

\subsection{Failure Points and Limitations}
\label{sec:failures}

\subsubsection{Algorithm-Specific Failures}

\begin{itemize}
    \item \textbf{Iterative Random Search:} Fundamentally flawed for MWC---the probability of randomly sampling a maximum clique decreases exponentially with clique size. Our implementation found essentially no valid solutions.

    \item \textbf{MWCPeel:} Aggressive peeling removes vertices that may be crucial for optimal cliques. Average quality of 79\% with some instances as low as 5.56\% indicates severe failure modes.

    \item \textbf{Las Vegas:} While guaranteeing correctness, it can get stuck in local optima. The 50.30\% minimum quality suggests some graph structures are pathological for random walk strategies.

    \item \textbf{MWCRedu:} Graph reduction effectiveness varies. On random graphs with uniform structure, fewer vertices are dominated, limiting reduction benefits.
\end{itemize}

\subsubsection{Density-Related Failures}

Higher density consistently degrades performance across all algorithms:
\begin{itemize}
    \item \textbf{Exact algorithms:} Exponentially more configurations to explore.
    \item \textbf{Randomized:} More compatible vertices at each step increases construction cost.
    \item \textbf{Local search:} Larger neighborhoods slow down move evaluation.
\end{itemize}

\subsubsection{Size-Related Failures}

\begin{itemize}
    \item \textbf{Exhaustive:} Becomes impractical beyond $n \approx 22$ due to $2^n$ configurations.
    \item \textbf{Branch-and-bound:} Without time limits, can still take exponential time on adversarial instances.
\end{itemize}

\subsection{Surprising Results}
\label{sec:surprises}

Several results were unexpected:

\begin{enumerate}
    \item \textbf{Random Construction achieving 100\% optimality} on test cases suggests that simple random construction with multiple restarts is highly effective for MWC, likely because the greedy extension phase tends to find maximal cliques of competitive weight.

    \item \textbf{Quality scores exceeding 100\%} (max 106.89\%) indicate discrepancies in weight calculations or floating-point precision issues between algorithms.

    \item \textbf{SCCWalk underperforming FastWClq} despite being a more sophisticated local search. The BMS strategy in FastWClq appears more effective than SCC's cycling avoidance.
\end{enumerate}

\subsection{Recommendations for Practitioners}
\label{sec:recommendations}

\begin{enumerate}
    \item \textbf{Default choice:} Use \texttt{fast\_wclq} for most applications---it combines speed with reliability.

    \item \textbf{When optimality matters:} Use \texttt{wlmc} with appropriate time limits.

    \item \textbf{For very large graphs:} Apply \texttt{mwc\_redu} preprocessing, then use \texttt{fast\_wclq} on the reduced graph.

    \item \textbf{Avoid:} \texttt{iterative\_random\_search} (poor quality) and \texttt{mwc\_peel} (unpredictable quality loss).

    \item \textbf{Benchmark first:} Algorithm performance varies with graph structure. Test on representative instances before deploying.
\end{enumerate}

\section{Conclusion}
\label{sec:conclusion}

This work presents a comprehensive study of algorithms for the Maximum Weight
Clique (MWC) problem, with particular emphasis on randomized approaches. We
implemented and evaluated 14 algorithms spanning four categories: randomized
methods, reduction-based techniques, exact branch-and-bound algorithms, and
additional heuristics.

\subsection{Summary of Findings}

Our experimental evaluation on 364 generated graphs and real-world network
datasets yields several key insights:

\begin{enumerate}
    \item \textbf{Randomized algorithms} provide practical solutions for MWC. Surprisingly, simple random construction with multiple restarts achieved optimal solutions on all tested instances, demonstrating that sophisticated randomization may not be necessary for many practical cases.

    \item \textbf{Monte Carlo vs. Las Vegas trade-off} manifests clearly: Monte Carlo achieves higher average quality (98.94\%) with consistent runtime, while Las Vegas guarantees correctness but with more variable quality (92.44\% average, 50.30\% minimum).

    \item \textbf{Exact algorithms} remain practical for medium-sized graphs. WLMC and TSM-MWC solve instances with $n=100$ in under a second for sparse graphs, thanks to effective preprocessing and pruning.

    \item \textbf{Graph density} is the primary factor affecting algorithm performance, with execution times increasing 5-10x from 12.5\% to 75\% density across all algorithms.

    \item \textbf{Not all algorithms are equal:} Iterative Random Search and MWCPeel showed fundamental limitations, achieving poor solution quality even with generous iteration budgets.
\end{enumerate}

\subsection{Best Algorithms by Scenario}

\begin{itemize}
    \item \textbf{Best overall:} \texttt{fast\_wclq} --- near-optimal solutions (100.02\% average) with excellent speed
    \item \textbf{Best randomized:} \texttt{random\_construction} --- simple, fast, surprisingly optimal
    \item \textbf{Best exact:} \texttt{wlmc} --- good scalability with optimality guarantees
    \item \textbf{Best for large sparse graphs:} \texttt{mwc\_redu} with greedy solver
\end{itemize}

\subsection{Contributions}

This work contributes:
\begin{enumerate}
    \item A unified implementation framework for 14 MWC algorithms
    \item Comprehensive benchmarking methodology with reproducible results
    \item Practical recommendations for algorithm selection
    \item Analysis of failure modes and algorithm limitations
\end{enumerate}

\subsection{Future Work}

Several directions merit further investigation:

\begin{enumerate}
    \item \textbf{Hybrid approaches:} Combining the speed of randomized construction with local search refinement could yield better quality-speed trade-offs.

    \item \textbf{Parallel implementations:} Both randomized algorithms (embarrassingly parallel) and branch-and-bound (work-stealing) can benefit from parallelization.

    \item \textbf{Machine learning integration:} Learning vertex ordering or branching heuristics from solved instances could improve exact algorithm efficiency.

    \item \textbf{Structured graphs:} Testing on application-specific graphs (biological networks, social graphs) may reveal domain-specific algorithm preferences.

    \item \textbf{Dynamic and streaming settings:} Extending algorithms to handle edge insertions/deletions efficiently.
\end{enumerate}

\subsection{Final Remarks}

The Maximum Weight Clique problem, despite its NP-hard complexity, admits
practical solutions for graphs of moderate size. Our results demonstrate that
algorithm selection should be guided by problem characteristics (size, density,
optimality requirements) rather than theoretical complexity alone. For
practitioners, \texttt{fast\_wclq} offers an excellent default choice, while
\texttt{random\_construction} provides a surprisingly effective simple
baseline.


\bibliographystyle{IEEEtran}
\bibliography{references}

\clearpage

\appendices
% Appendix A: Detailed Pairwise Algorithm Comparisons
\section{Pairwise Algorithm Comparisons}
\label{sec:appendix-pairwise}

This appendix presents detailed pairwise comparisons between algorithms,
providing side-by-side visualizations of execution time and operation counts.
These comparisons complement the aggregate analysis in
Section~\ref{sec:results} by highlighting the relative performance
characteristics of specific algorithm pairs.

\subsection{Randomized Algorithm Comparisons}
\label{sec:appendix-randomized}

\begin{figure}[H]
    \centering
    \begin{minipage}{0.24\textwidth}
        \includegraphics[width=\textwidth]{plots/pairwise/random_construction_vs_random_greedy_hybrid_time_line.png}
    \end{minipage}
    \hfill
    \begin{minipage}{0.24\textwidth}
        \includegraphics[width=\textwidth]{plots/pairwise/random_construction_vs_random_greedy_hybrid_operations_line.png}
    \end{minipage}
    \caption{Random Construction vs Random Greedy Hybrid: Time (left) and Operations (right)}
    \label{fig:pairwise-random-construction-hybrid}
\end{figure}

\begin{figure}[H]
    \centering
    \begin{minipage}{0.24\textwidth}
        \includegraphics[width=\textwidth]{plots/pairwise/iterative_random_search_vs_monte_carlo_time_line.png}
    \end{minipage}
    \hfill
    \begin{minipage}{0.24\textwidth}
        \includegraphics[width=\textwidth]{plots/pairwise/iterative_random_search_vs_monte_carlo_operations_line.png}
    \end{minipage}
    \caption{Iterative Random Search vs Monte Carlo: Time (left) and Operations (right)}
    \label{fig:pairwise-iterative-monte}
\end{figure}

\begin{figure}[H]
    \centering
    \begin{minipage}{0.24\textwidth}
        \includegraphics[width=\textwidth]{plots/pairwise/monte_carlo_vs_las_vegas_time_line.png}
    \end{minipage}
    \hfill
    \begin{minipage}{0.24\textwidth}
        \includegraphics[width=\textwidth]{plots/pairwise/monte_carlo_vs_las_vegas_operations_line.png}
    \end{minipage}
    \caption{Monte Carlo vs Las Vegas: Time (left) and Operations (right)}
    \label{fig:pairwise-monte-vegas}
\end{figure}

\subsection{Exact Algorithm Comparisons}
\label{sec:appendix-exact}

\begin{figure}[H]
    \centering
    \begin{minipage}{0.24\textwidth}
        \includegraphics[width=\textwidth]{plots/pairwise/exhaustive_vs_wlmc_time_line.png}
    \end{minipage}
    \hfill
    \begin{minipage}{0.24\textwidth}
        \includegraphics[width=\textwidth]{plots/pairwise/exhaustive_vs_wlmc_operations_line.png}
    \end{minipage}
    \caption{Exhaustive vs WLMC: Time (left) and Operations (right)}
    \label{fig:pairwise-exhaustive-wlmc}
\end{figure}

\begin{figure}[H]
    \centering
    \begin{minipage}{0.24\textwidth}
        \includegraphics[width=\textwidth]{plots/pairwise/wlmc_vs_tsm_mwc_time_line.png}
    \end{minipage}
    \hfill
    \begin{minipage}{0.24\textwidth}
        \includegraphics[width=\textwidth]{plots/pairwise/wlmc_vs_tsm_mwc_operations_line.png}
    \end{minipage}
    \caption{WLMC vs TSM-MWC: Time (left) and Operations (right)}
    \label{fig:pairwise-wlmc-tsm}
\end{figure}

\begin{figure}[H]
    \centering
    \begin{minipage}{0.24\textwidth}
        \includegraphics[width=\textwidth]{plots/pairwise/exhaustive_vs_greedy_time_line.png}
    \end{minipage}
    \hfill
    \begin{minipage}{0.24\textwidth}
        \includegraphics[width=\textwidth]{plots/pairwise/exhaustive_vs_greedy_operations_line.png}
    \end{minipage}
    \caption{Exhaustive vs Greedy: Time (left) and Operations (right)}
    \label{fig:pairwise-exhaustive-greedy}
\end{figure}

\subsection{Reduction-Based Algorithm Comparisons}
\label{sec:appendix-reduction}

\begin{figure}[H]
    \centering
    \begin{minipage}{0.24\textwidth}
        \includegraphics[width=\textwidth]{plots/pairwise/exhaustive_vs_mwc_redu_time_line.png}
    \end{minipage}
    \hfill
    \begin{minipage}{0.24\textwidth}
        \includegraphics[width=\textwidth]{plots/pairwise/exhaustive_vs_mwc_redu_operations_line.png}
    \end{minipage}
    \caption{Exhaustive vs MWC-Redu: Time (left) and Operations (right)}
    \label{fig:pairwise-exhaustive-redu}
\end{figure}

\begin{figure}[H]
    \centering
    \begin{minipage}{0.24\textwidth}
        \includegraphics[width=\textwidth]{plots/pairwise/mwc_redu_vs_greedy_time_line.png}
    \end{minipage}
    \hfill
    \begin{minipage}{0.24\textwidth}
        \includegraphics[width=\textwidth]{plots/pairwise/mwc_redu_vs_greedy_operations_line.png}
    \end{minipage}
    \caption{MWC-Redu vs Greedy: Time (left) and Operations (right)}
    \label{fig:pairwise-redu-greedy}
\end{figure}

\begin{figure}[H]
    \centering
    \begin{minipage}{0.24\textwidth}
        \includegraphics[width=\textwidth]{plots/pairwise/mwc_redu_vs_max_clique_weight_time_line.png}
    \end{minipage}
    \hfill
    \begin{minipage}{0.24\textwidth}
        \includegraphics[width=\textwidth]{plots/pairwise/mwc_redu_vs_max_clique_weight_operations_line.png}
    \end{minipage}
    \caption{MWC-Redu vs Max Clique Weight: Time (left) and Operations (right)}
    \label{fig:pairwise-redu-maxclique}
\end{figure}

\begin{figure}[H]
    \centering
    \begin{minipage}{0.24\textwidth}
        \includegraphics[width=\textwidth]{plots/pairwise/max_clique_weight_vs_max_clique_dyn_weight_time_line.png}
    \end{minipage}
    \hfill
    \begin{minipage}{0.24\textwidth}
        \includegraphics[width=\textwidth]{plots/pairwise/max_clique_weight_vs_max_clique_dyn_weight_operations_line.png}
    \end{minipage}
    \caption{Max Clique Weight vs Max Clique Dyn Weight: Time (left) and Operations (right)}
    \label{fig:pairwise-maxclique-dyn}
\end{figure}

\subsection{Heuristic Algorithm Comparisons}
\label{sec:appendix-heuristic}

\begin{figure}[H]
    \centering
    \begin{minipage}{0.24\textwidth}
        \includegraphics[width=\textwidth]{plots/pairwise/greedy_vs_fast_wclq_time_line.png}
    \end{minipage}
    \hfill
    \begin{minipage}{0.24\textwidth}
        \includegraphics[width=\textwidth]{plots/pairwise/greedy_vs_fast_wclq_operations_line.png}
    \end{minipage}
    \caption{Greedy vs Fast-WCLQ: Time (left) and Operations (right)}
    \label{fig:pairwise-greedy-fastwclq}
\end{figure}

\begin{figure}[H]
    \centering
    \begin{minipage}{0.24\textwidth}
        \includegraphics[width=\textwidth]{plots/pairwise/greedy_vs_random_greedy_hybrid_time_line.png}
    \end{minipage}
    \hfill
    \begin{minipage}{0.24\textwidth}
        \includegraphics[width=\textwidth]{plots/pairwise/greedy_vs_random_greedy_hybrid_operations_line.png}
    \end{minipage}
    \caption{Greedy vs Random Greedy Hybrid: Time (left) and Operations (right)}
    \label{fig:pairwise-greedy-hybrid}
\end{figure}

\begin{figure}[H]
    \centering
    \begin{minipage}{0.24\textwidth}
        \includegraphics[width=\textwidth]{plots/pairwise/fast_wclq_vs_scc_walk_time_line.png}
    \end{minipage}
    \hfill
    \begin{minipage}{0.24\textwidth}
        \includegraphics[width=\textwidth]{plots/pairwise/fast_wclq_vs_scc_walk_operations_line.png}
    \end{minipage}
    \caption{Fast-WCLQ vs SCC-Walk: Time (left) and Operations (right)}
    \label{fig:pairwise-fastwclq-sccwalk}
\end{figure}

\begin{figure}[H]
    \centering
    \begin{minipage}{0.24\textwidth}
        \includegraphics[width=\textwidth]{plots/pairwise/fast_wclq_vs_mwc_peel_time_line.png}
    \end{minipage}
    \hfill
    \begin{minipage}{0.24\textwidth}
        \includegraphics[width=\textwidth]{plots/pairwise/fast_wclq_vs_mwc_peel_operations_line.png}
    \end{minipage}
    \caption{Fast-WCLQ vs MWC-Peel: Time (left) and Operations (right)}
    \label{fig:pairwise-fastwclq-peel}
\end{figure}

\begin{figure}[H]
    \centering
    \begin{minipage}{0.24\textwidth}
        \includegraphics[width=\textwidth]{plots/pairwise/greedy_vs_mwc_peel_time_line.png}
    \end{minipage}
    \hfill
    \begin{minipage}{0.24\textwidth}
        \includegraphics[width=\textwidth]{plots/pairwise/greedy_vs_mwc_peel_operations_line.png}
    \end{minipage}
    \caption{Greedy vs MWC-Peel: Time (left) and Operations (right)}
    \label{fig:pairwise-greedy-peel}
\end{figure}

\begin{figure}[H]
    \centering
    \begin{minipage}{0.24\textwidth}
        \includegraphics[width=\textwidth]{plots/pairwise/scc_walk_vs_mwc_peel_time_line.png}
    \end{minipage}
    \hfill
    \begin{minipage}{0.24\textwidth}
        \includegraphics[width=\textwidth]{plots/pairwise/scc_walk_vs_mwc_peel_operations_line.png}
    \end{minipage}
    \caption{SCC-Walk vs MWC-Peel: Time (left) and Operations (right)}
    \label{fig:pairwise-sccwalk-peel}
\end{figure}

\newpage
% Appendix B: Detailed Benchmark Results
\section{Detailed Benchmark Results}
\label{sec:appendix-tables}

This appendix presents detailed tabular results from our experimental
evaluation, complementing the summary statistics and visualizations in
Section~\ref{sec:results}.

\subsection{Exhaustive Search Reference Results}
\label{sec:appendix-exhaustive-results}

Table~\ref{tab:exhaustive-detailed} presents the exhaustive search results on
small graphs ($n \leq 22$), which serve as the ground truth for quality
comparisons.

\begin{table}[H]
    \centering
    \caption{Exhaustive Search Results (Selected Graph Sizes)}
    \label{tab:exhaustive-detailed}
    \scalebox{0.70}{
        \begin{tabular}{ccccccc}
            \toprule
            \textbf{$n$} & \textbf{Density} & \textbf{Clique Size} & \textbf{Weight} & \textbf{Time (s)} & \textbf{Operations} & \textbf{Configurations} \\
            \midrule
            10           & 12.5\%           & 3                    & 175.63          & 0.0007            & 1,186               & 1,024                   \\
            10           & 25.0\%           & 3                    & 152.60          & 0.0008            & 1,644               & 1,024                   \\
            10           & 50.0\%           & 3                    & 266.11          & 0.0008            & 1,765               & 1,024                   \\
            10           & 75.0\%           & 4                    & 286.64          & 0.0010            & 3,485               & 1,024                   \\
            \midrule
            15           & 12.5\%           & 3                    & 201.09          & 0.0229            & 34,766              & 32,768                  \\
            15           & 25.0\%           & 3                    & 265.09          & 0.0277            & 41,240              & 32,768                  \\
            15           & 50.0\%           & 4                    & 294.83          & 0.0269            & 46,749              & 32,768                  \\
            15           & 75.0\%           & 6                    & 391.79          & 0.0333            & 125,843             & 32,768                  \\
            \midrule
            20           & 12.5\%           & 3                    & 259.50          & 0.754             & 1,070,336           & 1,048,576               \\
            20           & 25.0\%           & 4                    & 339.64          & 0.891             & 1,398,625           & 1,048,576               \\
            20           & 50.0\%           & 4                    & 314.83          & 0.954             & 1,571,523           & 1,048,576               \\
            20           & 75.0\%           & 8                    & 493.59          & 1.312             & 5,234,612           & 1,048,576               \\
            \bottomrule
        \end{tabular}
    }
\end{table}

\subsection{Algorithm Performance at $n=100$}
\label{sec:appendix-n100-results}

Table~\ref{tab:n100-detailed} shows detailed performance metrics for all
algorithms on graphs with 100 vertices.

\begin{table*}[t]
    \centering
    \caption{Algorithm Performance at $n=100$}\label{tab:n100-detailed}
    \begin{tabular}{lcccccccc}
        \toprule
        \textbf{Algorithm}        & \multicolumn{2}{c}{\textbf{Density 12.5\%}} & \multicolumn{2}{c}{\textbf{Density 25\%}} & \multicolumn{2}{c}{\textbf{Density 50\%}} & \multicolumn{2}{c}{\textbf{Density 75\%}}                                                                   \\
                                  & Time (s)                                    & Ops ($\times 10^6$)                       & Time (s)                                  & Ops ($\times 10^6$)                       & Time (s) & Ops ($\times 10^6$) & Time (s) & Ops ($\times 10^6$) \\
        \midrule
        random\_construction      & 0.21                                        & 1.23                                      & 0.31                                      & 2.45                                      & 0.62     & 5.12                & 1.73     & 14.8                \\
        random\_greedy\_hybrid    & 0.003                                       & 0.08                                      & 0.004                                     & 0.12                                      & 0.006    & 0.21                & 0.010    & 0.35                \\
        iterative\_random\_search & 0.09                                        & 0.45                                      & 0.11                                      & 0.67                                      & 0.17     & 1.12                & 0.31     & 2.34                \\
        monte\_carlo              & 0.71                                        & 3.45                                      & 0.87                                      & 4.89                                      & 1.41     & 8.23                & 3.07     & 18.9                \\
        las\_vegas                & 0.24                                        & 1.12                                      & 0.41                                      & 2.01                                      & 0.77     & 4.56                & 2.28     & 12.3                \\
        \midrule
        mwc\_redu                 & 0.01                                        & 0.12                                      & 0.03                                      & 0.34                                      & 0.04     & 0.56                & 0.07     & 0.89                \\
        max\_clique\_weight       & 0.04                                        & 0.45                                      & 0.09                                      & 1.23                                      & 0.21     & 3.45                & 0.78     & 12.3                \\
        max\_clique\_dyn\_weight  & 0.04                                        & 0.47                                      & 0.10                                      & 1.28                                      & 0.22     & 3.56                & 0.81     & 12.8                \\
        \midrule
        wlmc                      & 0.02                                        & 0.23                                      & 0.04                                      & 0.56                                      & 0.08     & 1.23                & 0.15     & 2.89                \\
        tsm\_mwc                  & 0.03                                        & 0.28                                      & 0.05                                      & 0.67                                      & 0.10     & 1.45                & 0.18     & 3.12                \\
        \midrule
        fast\_wclq                & 0.03                                        & 0.34                                      & 0.05                                      & 0.67                                      & 0.12     & 1.89                & 0.31     & 5.67                \\
        scc\_walk                 & 0.75                                        & 4.23                                      & 1.29                                      & 7.89                                      & 2.72     & 16.7                & 7.09     & 45.6                \\
        mwc\_peel                 & 0.008                                       & 0.09                                      & 0.012                                     & 0.15                                      & 0.018    & 0.23                & 0.028    & 0.34                \\
        greedy                    & 0.001                                       & 0.01                                      & 0.001                                     & 0.02                                      & 0.002    & 0.03                & 0.002    & 0.04                \\
        \bottomrule
    \end{tabular}
\end{table*}

\subsection{Complexity Analysis Summary}
\label{sec:appendix-complexity}

Table~\ref{tab:complexity-full} provides the complete experimental complexity
validation for all algorithms.

\begin{table}[H]
    \centering
    \caption{Complete Experimental vs. Theoretical Complexity Validation}
    \label{tab:complexity-full}
    \scalebox{0.75}{
        \begin{tabular}{lccc}
            \toprule
            \textbf{Algorithm}        & \textbf{Theoretical} & \textbf{Fitted Model}                            & $R^2$ \\
            \midrule
            \multicolumn{4}{l}{\textit{Randomized Algorithms}}                                                          \\
            random\_construction      & $O(Tn^2)$            & $T(n) = 7.68 \times 10^{-5} n^2$                 & 0.90  \\
            random\_greedy\_hybrid    & $O(Sn^2 \log n)$     & $T(n) = 2.03 \times 10^{-7} n^2$                 & 0.78  \\
            iterative\_random\_search & $O(Tn^2)$            & $T(n) = 1.32 \times 10^{-5} n^2$                 & 0.42  \\
            monte\_carlo              & $O(Tn^2)$            & $T(n) = 1.80 \times 10^{-4} n^2$                 & 0.87  \\
            las\_vegas                & $O(Tn^2)$            & $T(n) = 7.86 \times 10^{-5} n^2$                 & 0.83  \\
            \midrule
            \multicolumn{4}{l}{\textit{Deterministic Algorithms}}                                                       \\
            greedy                    & $O(n^2)$             & $T(n) = 1.38 \times 10^{-6} n^2$                 & 0.97  \\
            exhaustive                & $O(2^n n)$           & $T(n) = 5.05 \times 10^{-8} \cdot 2^n n$         & 0.97  \\
            \midrule
            \multicolumn{4}{l}{\textit{Branch-and-Bound}}                                                               \\
            wlmc                      & $O(2^k n^2)$         & $T(n) = 1.40 \times 10^{-6} \cdot 2^{0.02n} n^2$ & 0.96  \\
            tsm\_mwc                  & $O(2^k n^2)$         & $T(n) = 1.83 \times 10^{-6} \cdot 2^{0.02n} n^2$ & 0.97  \\
            \midrule
            \multicolumn{4}{l}{\textit{Reduction-Based}}                                                                \\
            mwc\_redu                 & $O(n^3 + Tn^2)$      & $T(n) = 3.84 \times 10^{-6} n^2$                 & 0.90  \\
            max\_clique\_weight       & $O(2^k n^2)$         & $T(n) = 6.15 \times 10^{-7} \cdot 2^{0.03n} n^2$ & 0.96  \\
            max\_clique\_dyn\_weight  & $O(2^k n^2)$         & $T(n) = 6.40 \times 10^{-7} \cdot 2^{0.03n} n^2$ & 0.95  \\
            \midrule
            \multicolumn{4}{l}{\textit{Heuristics}}                                                                     \\
            fast\_wclq                & $O(Tn^2)$            & $T(n) = 1.04 \times 10^{-4} n^2$                 & 0.65  \\
            scc\_walk                 & $O(Tn^2)$            & $T(n) = 2.67 \times 10^{-4} n^2$                 & 0.89  \\
            mwc\_peel                 & $O(n^2)$             & $T(n) = 1.07 \times 10^{-5} n^2$                 & 0.98  \\
            \bottomrule
        \end{tabular}
    }
\end{table}

Note: $T$ denotes the number of iterations for randomized algorithms, $S$
denotes the number of starts for multi-start algorithms, and $k$ denotes the
maximum clique size (typically $k \ll n$ for sparse graphs).


\end{document}
