\section{O Problema}\label{sec:problema}

Seja $G=(V,E)$ um grafo não direcionado com $n=|V|$ vértices e $m=|E|$ arestas.
Cada vértice $v \in V$ possui um peso positivo $w(v) > 0$ atribuído através de uma função $w: V \to \mathbb{R}^+$.
\begin{definition}[Clique]
Uma \textbf{clique} é um subconjunto de vértices $C \subseteq V$ tal que todos os vértices em $C$ são adjacentes entre si, ou seja, o subgrafo induzido por $C$ é completo.
Formalmente, $C$ é uma clique se e somente se:
$$\forall u, v \in C, u \neq v: \{u,v\} \in E$$
\end{definition}

\begin{definition}[Peso de uma Clique]
O \textbf{peso de uma clique} $C$ é definido como a soma dos pesos dos seus vértices:
$$w(C) = \sum_{v \in C} w(v)$$
\end{definition}

\begin{definition}
Dado um grafo não direcionado $G(V,E)$ com pesos positivos nos vértices, o problema \textbf{Maximum Weight Clique} consiste em encontrar um clique $C^*$ que maximize o peso total:
$$C^* = \arg\max_{\substack{C \subseteq V \\ C \text{ é clique}}} w(C)$$
\end{definition}

\subsection{Propriedades do Problema}

O problema Maximum Weight Clique é um problema de otimização combinatória que apresenta as seguintes características:

\begin{itemize}
    
\item \textbf{NP-dificuldade}: O problema é NP-difícil \cite{garey1979}, pois a versão de decisão (dado um grafo e um valor $k$, existe uma clique com peso pelo menos $k$?) é NP-completa.
Isto implica que não se conhece um algoritmo polinomial que resolva o problema de forma ótima para instâncias arbitrárias.
\item \textbf{Complexidade do espaço de procura}: Para um grafo com $n$ vértices, existem $2^n$ possíveis subconjuntos de vértices a considerar, tornando a procura exaustiva impraticável para valores grandes de $n$.
\item \textbf{Pesos positivos}: Neste trabalho, consideramos apenas pesos positivos nos vértices.
A extensão para pesos negativos ou nulos introduz complexidades adicionais.
\item \textbf{Relacionamento com problemas clássicos}: O problema Maximum Weight Clique generaliza o problema clássico de clique máxima (onde todos os pesos são unitários), sendo mais complexo e com aplicações mais amplas.
\end{itemize}

O problema encontra aplicações práticas em diversas áreas. Na computação social, é usado para a identificação de comunidades coesas e influentes em \textit{big data}. Na bioinformática, é fundamental para a análise de redes de interação proteica, descoberta de módulos funcionais e, especificamente, na análise de locais de ligação de proteínas \cite{konc2025efficient} e na predição de complexos de RNA \cite{rcpred2019}.