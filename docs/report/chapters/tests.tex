\section{Testes Computacionais}\label{sec:testes}

Esta secção apresenta a metodologia experimental utilizada, os resultados obtidos e a comparação entre a análise formal de complexidade (apresentada na Secção~\ref{sec:algoritmos}) e os resultados experimentais observados.
\subsection{Metodologia Experimental}

\subsubsection{Geração de Grafos}

Os grafos experimentais foram gerados utilizando um gerador aleatório com as seguintes características:

\begin{itemize}
  \item \textbf{Vértices:} Pontos 2D com coordenadas inteiras uniformemente distribuídas entre 1 e 500
  \item \textbf{Distância mínima:} $\geq 10$ unidades entre vértices para evitar sobreposição espacial
  \item \textbf{Pesos:} Valores aleatórios uniformemente distribuídos entre 1.0 e 100.0, garantindo que todos os pesos são positivos
  \item \textbf{Arestas:} Seleccionadas aleatoriamente para atingir densidades específicas
  \item \textbf{Densidades testadas:} 12.5\%, 25\%, 50\% e 75\% do número máximo possível de arestas ($\binom{n}{2}$)
  \item
        \textbf{Tamanhos:} De 4 a 20 vértices para comparação directa entre ambos os algoritmos;
        até 30 vértices para análise de escalabilidade da heurística
  \item \textbf{Seed:} 112974 para garantir reprodutibilidade dos resultados
\end{itemize}

A Figura~\ref{fig:example_graph} ilustra um exemplo de grafo gerado com 18 vértices e densidade de 50\%, onde o clique de peso máximo encontrado está destacado a vermelho.

\begin{figure}[h!]
  \centering
  \includegraphics[width=0.48\textwidth]{../../experiments/results/visualization_max_w_clique_n18_d50.png}
  \caption{Exemplo de grafo gerado (18 vértices, densidade 50\%) com clique de peso máximo destacado}
  \label{fig:example_graph}
\end{figure}

\subsubsection{Métricas Colectadas}

Para cada algoritmo e instância de grafo, foram colectadas as seguintes métricas:

\begin{enumerate}
  \item \textbf{Tempo de execução:} Tempo real medido usando \texttt{time.perf\_counter()} em segundos
  \item \textbf{Operações básicas:} Número de verificações de adjacência realizadas (métrica independente do hardware)
  \item \textbf{Configurações testadas:} Número de subconjuntos de vértices examinados durante a execução
  \item \textbf{Precisão da heurística:} Percentagem de qualidade relativa ao ótimo: $\frac{peso\_guloso}{peso\_ótimo} \times 100\%$
  \item \textbf{Factor de aceleração:}
        Razão entre tempos de execução: $\frac{tempo\_exaustivo}{tempo\_guloso}$
\end{enumerate}

\subsubsection{Ambiente Experimental}

\begin{itemize}
  \item \textbf{Sistema operativo:} Linux 6.17.0-6-generic
  \item \textbf{Linguagem:} Python 3.14
  \item \textbf{Reprodutibilidade:} Seed fixa (NMEC) para geração de grafos
\end{itemize}

\subsection{Resultados Experimentais}

Nesta secção é apresentada uma \textbf{seleção representativa} dos resultados obtidos, focando em grafos com tamanhos de 10, 15 e 20 vértices, que permitem comparação direta entre ambos os algoritmos e demonstram claramente as diferenças de desempenho.
Para cada tamanho, estão incluídos exemplos das quatro densidades testadas.

\subsubsection{Análise de Performance Temporal}

A Figura~\ref{fig:execution_time} mostra o tempo de execução em função do número de vértices para ambos os algoritmos e diferentes densidades.
\begin{figure}[h!]
  \centering
  \includegraphics[width=0.48\textwidth]{../../experiments/plots/execution_time.png}
  \caption{Tempo de execução vs. número de vértices para ambos os algoritmos}
  \label{fig:execution_time}
\end{figure}

Observa-se claramente o crescimento exponencial do algoritmo exaustivo em contraste com o crescimento polinomial da heurística gulosa.
Para grafos com mais de 15 vértices, o algoritmo exaustivo torna-se impraticável, enquanto a heurística mantém tempos de execução muito baixos.
\subsubsection{Contagem de Operações Básicas}

A Figura~\ref{fig:operations_count} apresenta o número de operações básicas (verificações de adjacência) executadas pelo algoritmo exaustivo.
\begin{figure}[h!]
  \centering
  \includegraphics[width=0.48\textwidth]{../../experiments/plots/operations_count.png}
  \caption{Número de operações básicas vs. número de vértices (algoritmo exaustivo)}
  \label{fig:operations_count}
\end{figure}

Este gráfico confirma a análise teórica: o algoritmo exaustivo executa um número exponencial de operações, com crescimento acelerado à medida que o número de vértices aumenta.
\subsubsection{Configurações Testadas}

A Figura~\ref{fig:configurations_tested} ilustra o crescimento exponencial $2^n$ das configurações testadas pelo algoritmo exaustivo.
\begin{figure}[h!]
  \centering
  \includegraphics[width=0.48\textwidth]{../../experiments/plots/configurations_tested.png}
  \caption{Número de configurações testadas vs. número de vértices (algoritmo exaustivo)}
  \label{fig:configurations_tested}
\end{figure}

Observa-se que o número de configurações testadas segue exactamente a função $2^n$, confirmando que o algoritmo testa todos os subconjuntos possíveis de vértices.
\subsubsection{Qualidade da Heurística}

A Figura~\ref{fig:heuristic_precision} mostra a precisão da heurística gulosa em relação à solução ótima.
A robustez da heurística deve ser testada em instâncias SOTA (e.g., grafos de bioinformática \cite{konc2025efficient} ou map labeling \cite{erhardt2023improved}), não apenas em grafos aleatórios maiores.

\begin{figure}[h!]
  \centering
  \includegraphics[width=0.48\textwidth]{../../experiments/plots/heuristic_precision.png}
  \caption{Precisão da heurística gulosa vs. número de vértices}
  \label{fig:heuristic_precision}
\end{figure}

A heurística mantém alta precisão (geralmente acima de 94\%) em relação à solução ótima, demonstrando que a estratégia multi-início é eficaz na obtenção de soluções de qualidade.

A Figura~\ref{fig:greedy_larger_n} apresenta o tempo de execução da heurística gulosa para grafos com até 500 vértices. Esta figura demonstra claramente o comportamento esperado do algoritmo para instâncias de maior dimensão. Onde o tempo aumenta de forma polinomial.

\begin{figure}[h!]
  \centering
  \includegraphics[width=0.48\textwidth]{../../experiments/plots.500/execution_time_heuristic.png}
  \caption{Tempo de execução da heurística gulosa para grafos até $n=500$ vértices}
  \label{fig:greedy_larger_n}
\end{figure}

\subsection{Análise Detalhada dos Resultados}

A Tabela~\ref{tab:results_summary} apresenta uma seleção representativa dos resultados experimentais para diferentes tamanhos de grafo.
Esta tabela resume os dados mais relevantes que demonstram as diferenças de desempenho entre os algoritmos.
Os resultados completos de todos os benchmarks realizados (68 entradas) encontram-se no Anexo \ref{attach:complete_results} (Tabela~\ref{tab:complete_results}).


\begin{table}[h!]
  \centering
  
  \label{tab:results_summary}
  \small
  \begin{tabular}{@{}ccccccccc@{}}
    \toprule
    V  & E   & $\rho$ & Ex.(s) & Gu.(s) & Pr.(\%) & Op.Ex.    & Op.Gu. \\
    \midrule
    10 & 5   & 12.5   & .0008 & 1e-4    & 100     & 1186     & 168     \\
    10 & 11  & 25.0   & .0007 & 1e-4    & 100     & 1644     & 215     \\
    10 & 22  & 50.0   & .0008 & 1e-4    & 100     & 1765     & 338     \\
    10 & 33  & 75.0   & .0014 & 2e-4    & 100     & 3485     & 524     \\
    15 & 13  & 12.5   & .0246 & 1e-4    & 100     & 34766    & 422     \\
    15 & 26  & 25.0   & .0250 & 2e-4    & 100     & 41240    & 576     \\
    15 & 52  & 50.0   & .0260 & 2e-4    & 100     & 46749    & 1005   \\
    15 & 78  & 75.0   & .0327 & 4e-4    & 100     & 125843   & 2116   \\
    20 & 23  & 12.5   & .8072 & 2e-4    & 100     & 1116260 & 877     \\
    20 & 47  & 25.0   & .8010 & 3e-4    & 81.2      & 1613817 & 957     \\
    20 & 95  & 50.0   & .9002 & 4e-4    & 100     & 2526647 & 2205   \\
    20 & 142 & 75.0   & .9021 & 9e-4    & 100     & 2915493 & 5220   \\
    \bottomrule
  \end{tabular}
  \vspace{1em}
  \caption{Resumo dos Resultados Experimentais (Selecção Representativa).\\Legenda: V (Vértices), E (Arestas), $\rho$ (Densidade), Ex.(s) (Tempo Exaustivo), Gu.(s) (Tempo Guloso), Pr.(\%) (Precisão Heurística), Op.Ex. (Operações Exaustivo), Op.Gu. (Operações Guloso).}
  \vspace{-1em}
\end{table}


Com esta tabela conseguimos verificar tudo o que foi discutido anteriormente e confirmar que a densidade do grafo não representa um \textit{bottleneck}, tanto para o tempo de execução como para o número de operações.

\subsection{Comparação entre Análise Formal e Experimental}

\subsubsection{Validação da Complexidade do Algoritmo Exaustivo}

A análise teórica prevê complexidade temporal $O(2^n \times n^2)$ e espacial $O(n)$.
Os resultados experimentais confirmam este comportamento:

\begin{itemize}
  \item \textbf{Configurações testadas:} Os dados experimentais (Figura~\ref{fig:configurations_tested}) mostram que o número de configurações testadas segue exatamente $2^n$, confirmando a enumeração de todos os subconjuntos. (e.g., para $n=10$, $2^{10} = 1024$; para $n=20$, $2^{20} = 1.048.576$).

  \item \textbf{Crescimento do tempo:} O tempo de execução dobra aproximadamente a cada vértice adicional, confirmando o crescimento exponencial $O(2^n)$.
  \item \textbf{Operações básicas:} O número de operações (Tabela~\ref{tab:results_summary}, col. "Op.Ex.") cresce exponencialmente, sendo consistente com a análise teórica de $O(2^n \times n^2)$.
\end{itemize}

\subsubsection{Validação da Complexidade da Heurística Gulosa}

A análise teórica prevê complexidade temporal $O(n^4)$ (pior caso) e espacial $O(n)$.
Os resultados experimentais confirmam comportamento polinomial:

\begin{itemize}
  \item \textbf{Tempo de execução:} Cresce de forma polinomial, muito mais lento que o crescimento exponencial do algoritmo exaustivo.
  \item \textbf{Operações básicas:} O número de operações (Tabela~\ref{tab:results_summary}, col. "Op.Gul.") cresce polinomialmente, sendo consistente com a análise teórica.
  \item \textbf{Escalabilidade:} A heurística pode processar grafos com centenas de vértices em tempo razoável, enquanto o algoritmo exaustivo torna-se impraticável acima de 20 vértices.
\end{itemize}



Para quantificar o crescimento, calculou-se a média das operações da heurística para cada tamanho de grafo (considerando as diferentes densidades).

Com os valores obtidos, ajustou-se uma regressão $log-log$ segundo o modelo $Op = c \cdot V^p$, obtendo-se:

\[
  p \approx 2.42 \quad \text{e} \quad c \approx 1.69
\]

O expoente ajustado $p$ aproxima-se de $3$, à medida que o número de vértices aumenta, indicando que o crescimento é cúbico, ou seja, o número de operações cresce aproximadamente como $O(n^3)$.

\begin{figure}[h!]
  \centering
  \includegraphics[width=0.48\textwidth]{../../experiments/plots.500/loglog_greedy_fit.png}
  \caption{Ajuste $log-log$ das operações da heurística gulosa ($p \approx 2.42$).}
  \label{fig:loglog_greedy_fit}
\end{figure}

A Figura~\ref{fig:loglog_greedy_fit} ilustra o ajuste $log-log$ com linha de regressão linear, confirmando empiricamente o comportamento cúbico do algoritmo. Assim, tanto a evidência experimental como a análise teórica sustentam a conclusão de que a heurística apresenta complexidade $O(n^3)$.

\subsubsection{Factores de Aceleração}

A Tabela~\ref{tab:speedup} apresenta os factores de aceleração da heurística gulosa em relação ao algoritmo exaustivo.
\begin{table}[h!]
  \centering
  \caption{Factores de Aceleração da Heurística Gulosa}
  \label{tab:speedup}
  \small
  \begin{tabular}{@{}ccc@{}}
    \toprule
    Vértices & Factor de Aceleração & Redução de Operações \\
    \midrule
    10       & 9.1x                 & 84.6\%               \\
    15       & 124.8x               & 98.4\%               \\
    20       & 2,449.6x             & 99.9\%               \\
    \bottomrule
  \end{tabular}
\end{table}

O factor de aceleração aumenta exponencialmente com o número de vértices, demonstrando que a vantagem da heurística cresce drasticamente para grafos maiores.
\subsubsection{Impacto da Densidade}

A densidade do grafo afecta o desempenho de ambos os algoritmos:

\begin{itemize}
  \item \textbf{Grafos densos:} Maior número de cliques possíveis, aumentando o número de operações em ambos os algoritmos.
  \item \textbf{Grafos esparsos:} Menor número de cliques, reduzindo o espaço de busca.
  \item \textbf{Precisão da heurística:} Mantém-se elevada (geralmente acima de 94\%) independentemente da densidade, com média geral superior a 99\% nos benchmarks testados.
\end{itemize}

\subsection{Síntese dos Resultados Experimentais}

Os resultados experimentais validam completamente as análises teóricas de complexidade apresentadas na Secção~\ref{sec:algoritmos}:

\begin{enumerate}
  \item O algoritmo exaustivo apresenta crescimento exponencial confirmado em tempo, operações e configurações testadas ($2^n$).
  \item A heurística gulosa apresenta crescimento polinomial confirmado, mantendo tempos de execução muito baixos mesmo para grafos maiores.
  \item A qualidade das soluções da heurística é excelente nos grafos aleatórios testados, com precisão média superior a 99\%.
  \item O factor de aceleração da heurística aumenta exponencialmente com o tamanho do grafo.
  \item A densidade do grafo tem impacto moderado no desempenho, mas não afecta significativamente a qualidade das soluções da heurística.
\end{enumerate}

A concordância entre análise teórica e resultados experimentais demonstra a correcção das análises formais e a eficácia da implementação dos algoritmos baseline.